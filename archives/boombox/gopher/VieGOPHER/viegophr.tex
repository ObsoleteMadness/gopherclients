
%
% File viegophr.tex
%
%+++++++++++++++++++++++++++++++++++++++++++++++++++++++++++++++++++++++++
%+ INTERNET GOPHER, CLIENT & SERVER FOR IBM VM/CMS    (1992-06-19/15:00) +
%+                                                                       +
%+ Notes on installation and usage of the `Vienna' VM/CMS GOPHER System. +
%+++++++++++++++++++++++++++++++++++++++++++++++++++++++++++++++++++++++++
%
% NOTE: This is not the original version of the document!
%       Instead, this is the sequenced and stripped version of a
%       hypertext file.
%       The linear LaTeX version is provided for your convenience.
%       For more information about the hypertext version contact the author.
%       LaTeX is described in: (I'm not sure..)
%         Leslie Lamport:  LaTeX, A document preparation system
%         Addison Wesley
%
\documentstyle{article}
\pagestyle{headings}

% pagelayout ---------
% \topmargin10mm
% \headheight12mm
% \headsep5mm
% \topskip0mm
% \textheight245mm
% \textwidth175mm
% \oddsidemargin17mm
% \evensidemargin65mm
% \marginparwidth46mm
% \marginparsep5mm
% \marginparpush5mm

%needed???? yes...
\newfont{\vsmtt}{cmtt8 scaled 1000} % very small type writer

%\makeindex

\begin{document}



\pagenumbering{roman}
\thispagestyle{empty}
\title{VieGOPHER\\---\\A Gopher System for VM/CMS}
\author{Gerhard Gonter}
\date{\today}
\maketitle

\begin{abstract}\footnote{This
file is still somewhat ... er ... incomplete ... sorry ...}
The Internet Gopher Protocol provides a simple, yet very effective
method for distributed document search and retrieval.  This document
describes usage, installation, maintainance and also some points 
about the implementation of a Gopher
system for IBM's operating system VM/CMS, called the VieGOPHER.
Both, using the client and administrating the server are discussed.

The system can be distributed freely and is dedicated to the community
of cyberspace.
\end{abstract}

\pagebreak[4]
\tableofcontents
\pagebreak[4]
\listoftables
\listoffigures
\pagebreak[4]

\newcommand{\SUBsection}[1]{\subsection{#1}}
\newcommand{\SUBsubsection}[1]{\subsubsection{#1}}

% ---------------------------------------------------------------------- 
\setcounter{page}{1}
\pagenumbering{arabic}



\section{Introduction}

This document describes an implementation of the Internet GOPHER Server
and Client protocol for IBM's operating system VM/CMS, known as the
`Vienna' VM Gopher System or `VieGOPHER'.
There are things left out and enhancements will follow.  If you find any
errors or have other suggestion then don't hesitate to contact me on one
of the addresses given in section \ref{gg.addr} on page \pageref{gg.addr}.
 
The system consists of a client, based on {\sf XEDIT} ({\tt GOPHERC.EXEC}
together with
a bunch of {\tt .XEDIT} and {\tt .EXEC} files) and two servers,
the older one is a simple,
straight forward, plain vanilla, nothing special, ordinary, single-treaded
Internet Gopher Server\footnote{not supported anymore} ({\tt GOPHERD.EXEC}),
the new one ({\tt GOPHERDD.EXEC}) %\index{GOPHERDD.EXEC}
is just the opposite of that all...  well...  the most interesting feature
of it is that it is multi-threaded.  This means, it can handle more than
one connection at the same time and, in fact, the version I currently run
listens on four ports at the same time and is still able to handle a
couple of connections simultanously.  Since this server listens on four
different ports you might suspect that one is the officially assigned
gopher port (70) and the the other two ports are something else.  This
assumption is not completely wrong, the server also listens on port 79
and pretends to be a finger server as well on port 17 and serves out fortune
cookies.  The service on port 17 is also known as the Quote-of-the-Day
protocol.  Finger protocol (see \cite{rfc1288})
and Quote-of-the-Day protocol (see \cite{rfc865})
are handled by the Gopher Server as a by-product,
as you may have time to read in the code.
 
In order to run either the server or the client you also need a program
called {\tt REXTCPIP.MODULE} (by Ken Hornstein).  You can find it on any
{\sf BITNET LISTSERV} that carries {\sf VM-UTIL}, (e.g. {\sf DEARN}).
Be sure to use version 1.8.4. or newer.  Section \ref{bugs} gives
information about modifications that improve the stability of the
program.  If necessary, a modified version can be obtained from me.
You also need TCPIP to run either the server or the client.
The new server can also employ {\tt YWAKEUP.MODULE} to trap messages
from servicing machines.

Clients running under TCPIP version 2 can take advantage of RXSOCKET
instead of REXTCPIP.  RXSOCKET is generally more elaborate and
stable, either version 1 or can be used.

The instability of REXTCPIP is the main weaknes of the server.
To overcome this problem I work along three paths:
\begin{enumerate}
\item   I try to make the REXX code of the server as shock-proof
        as possible,
\item   I try to correct some of the problems inherent in
        REXTCPIP, see section \ref{bugs} for a documentation
        about some of the patches sofar,
\item   allowing to select RXSOCKET seems to be the most
        promising path in the long run.
\end{enumerate}
 


\subsection{Features}

\noindent Client:
\noindent\begin{itemize}
\item can work with WAIS search gateways
\item convenient XEDIT interface
\end{itemize}

\noindent Server:
\noindent\begin{itemize}
\item multiple connections
\item integrated finger server
\item integrated cookie server
\item interface for embedded applications
\end{itemize}

% ---------------------------------------------------------------------- 


\subsection{{\sc Rose} installation procedure}

The source code of the programs and the parts of the \LaTeX\ documentation
are stored together in NED's HYX format --- NED is a MSDOS `hypertexty'
text editor.  From this source three text files are generated:
\begin{itemize}
\item   {\tt VIEGOPHR.PRO(DUCT)} containing the program code,
\item   {\tt VIEGOPHR.TEX} containing the documentation,
        (this document)
\item   {\tt VIEG-PRO.TEX} containing the source code listing.
\end{itemize}

{\tt VIEGOPHR.PRODUCT} needs to be processed by {\tt ROSE.EXEC}.
{\sc Rose} is not part of VieGOPHER, however, it was developed
for it's installation.
The program can be distributed freely just as the other
package.  

% To install VieGOPHER using {\sc Rose} these conditions should be met:
% $\Box$ the files {\tt ROSE.EXEC} and {\tt VIEGOPHR.PRODUCT} in
% read access on any minidisk.  A read/write minidisk as A with at
% least XXX free blocks.  If you have the $\Box$ optional file
% {\tt VIEGOPHR.FIX}, possibly from an earlier experience with {\sc Rose},
% then you should place it now on minidisk A.  Note:  This file
% will be changed or created during installation.

\subsubsection{Quick'n'dirty installation}

Appendix \ref{inst.opts} on page \pageref{inst.opts} lists all {\sc Rose}
prompt texts that will be presented during the installation phase.

\noindent
Use this checklist to get the system off the ground:
\begin{itemize}
\item[1 $\Box$] Put {\tt VIEGOPHR.PRODCUT} and {\tt ROSE.EXEC}
                on any readable minidisk.

\item[2 $\Box$] Prepare a R/W minidisk with filemode A with at
                least 2 cylinders (275 blocks with 4096 byte).

\item[3 $\Box$] If you have {\tt VIEGOPHR.FIX} around somewhere,
                you can place it on minidisk A, but this is not necessary.

\item[4 $\Box$] Start the program generator by entering:
                {\tt ROSE VIEGOPHR}

\item[5 $\Box$]  Read the options presented by {\sc Rose} and answer the questions
        appropriately.  In most cases you will want to use the
        predefined option.  In this case simply enter code 1 or hit
        the ENTER key.  If you want to change something enter code 3
        (followed by pressing the ENTER key of course) and then enter
        the value that you want to use.

\item[6 $\Box$]  After prompting for all options {\sc Rose} will generate all
        modules and invoke the postinstallation procedure {\tt GOPHINST.EXEC}.

\item[7 $\Box$]  Fetch any missing file that the postinstallion procedure might
        point out.  If the postinstallation procedure finds
        inconstencies in your installation, remove the appropriate
        line from {\tt VIEGOPHR.FIX} and go back to step 4.

\item[8 $\Box$]  Put the file {\tt VIEGOPHR.FIX} at a safe place.  You might want
        to use it again for a fast re-installation later on.

\item[9 $\Box$]  Place {\tt RXSOCKET.MODULE} and/or {\tt REXTCPIP.MODULE} on minidisk A.

\item[10 $\Box$]  Done!  Your minidisk A should now contain all necessary programs 
        for the client and server.

\item[11 $\Box$]  If you don't want the server you can remove {\tt GOPHERDD.EXEC}.

\item[12 $\Box$]  If you want to run the server it's time to look through
        this document ({\tt VIEGOPHR.PS}) now to find out about
        index files, access groups, etc...
\end{itemize}



\section{Client}

% * Client Modules
% * General Module Layout
%   * Modules from other Sources
% * Installation
%   * Quick Installation
%     Professional Installation
%     * (1)
%     * Fig: Instsec in GOPHER.EXEC
%     * (2)
%     * Fig: Instsec in GOPHFTCH.EXEC
%     * (3)
% * Files
% * Global Variables
%   * User Setup
%   * temporary variables used by the GOPHER modules
% * Using the Client
%   * Fig: Root Index Screen
%   * temp files



\subsection{Client Modules}

The client is based on {\sf XEDIT} and {\sf REXTCPIP} or {\sf RXSOCKET}.
It does not
require any other non-standard extensions of the operating system, like
{\sf CMS-PIPES} and should also work with TCP/IP release 1 (the
REXTCPIP version only).




\subsection{General Module Layout}

 


\subsubsection{Modules from other Sources}
 
The files {\tt REXTCPIP.MODULE} and {\tt RXSOCKET.MODULE}
alternatively provide the network transport system.
 
TCP/IP must be available to the user; the TCPIP public disk is not
required unless the user also wants to use TELNET.
 
 


\subsection{Installation}

%   * Quick Installation
%   * Professional Installation
%     * (1)
%     * Fig: Instsec in GOPHER.EXEC
%     * (2)
%     * Fig: Instsec in GOPHFTCH.EXEC
%     * (3)


\subsubsection{Quick Installation}

Installing the VieGOPHER client quickly is fairly easy:  Just get the
distribution package\footnote{see appendix \ref{sources} how to obtain it},
and follow the installation procedures given in the readme file (or
in the introduction),
get {\tt REXTCPIP.MODULE} version 1.8.4 or {\tt RXSOCKET.MODULE},
put everything on your minidisk, and type {\tt GOPHER}.
If everything is ok you should get the root menu of the server
{\tt gopher.wu-wien.ac.at}\footnote{Figure \ref{run.cl1} on page
\pageref{run.cl1} shows such a menu}, otherwise you have a problem.
% and you should consider reading the upcomming section about
% professional installation.



\SUBsection{Using the Client}

The gopher client is invoked by executing GOPHER.  If everything is set up
properly, the root menu of your default server should appear within a
few seconds.  Figure \ref{run.cl1} shows a root menu.  The beginning of
a menu line contains an item marker, {\tt $<$F$>$} indicates a text file
and {\tt $<$D$>$} indicates another menu or directory.
Table \ref{supported.items} lists all item types that are supported
by this client.  Please note that other servers might also provide
different item types, these are flagged by a marker of the form
{\tt $>$}{\sl{x}}{\tt{$<$}}.  A list of other know item types is provided
in table \ref{supported.items}.

%    * Fig: Root Index Screen
%    * Programmable Function Keys
%      * (1)
%      * Fig: Programmable Function Keys
%      * (2)
%    * Bookmarks
%    * saving items
%    * Feedback
%    * Outlook
%    * temp files
%    * Supported Item Types
%    * Tbl: Supported Item Types


\begin{figure}
\caption{Root Index of a Gopher Server}\label{run.cl1}
\rule{0mm}{2mm}
\begin{center}
\begin{small}
\begin{verbatim}
 GOPHTMP  TMP0     A1  V 122  Trunc=122 Size=20 Line=1 Col=1 Alt=0
====>
<D> WU-Wien Gopher Information Service: Main Menu
gopher.wu-wien.ac.at:70 <1>                                                  
<F> About Internet Gopher (gopher.wu-wien.ac.at:70)
<D> Local Info
<F> ----------------------------------------------------------------------
<D> Gophers around the house (Univ. Economics, Vienna)
<D> Gophers around the place (Austrian Universities)
<D> Gophers around the world (and other information services)
<F> ----------------------------------------------------------------------
<D> Early Music List: archives of EARLYM-L@AEARN.bitnet + rec.music.early
<D> CERRO: the Central European Regional Research Organization
<D> WAFE: Widget Athena Frontend
<D> documents
<D> services
<D> external services
<D> pointers                        (things you might also find elsewhere)
<D> life, the NET and everything
<D> The Guide!
<F> ----------------------------------------------------------------------
<D> VieGOPHER: VM/CMS Gopher Version
 1= Help      2=(altpfs)  3= GoBack  4= Disp->BMK   5= Curs->BMK   6= BMK  
 7= PageUP    8= PageDN   9= Save   10=            11= Display    12= Circle


\end{verbatim}
\end{small}
\end{center}
\rule{0mm}{2mm}
\end{figure}
 


\SUBsubsection{Programmable Function Keys}

Basically, the user interface is `pure' {\sf XEDIT} with a customized
PF-key setting.  How the keys are set up is shown in table \ref{PF.keys}.



\begin{table}
\caption{Programmable Function Keys}\label{PF.keys}
\rule{0mm}{2mm} % see also * setup for the keys
\begin{center}
\noindent\begin{tabular}{|l|l|l|}
\hline
Key     & Label             & Description\\
\hline\hline
PF1     & {\tt Help}        & activate help menu (just for XEDIT)           \\
PF2     & {\tt (altpfs)}    & assing PF 1-12 with functions of PF 13-24     \\
PF3     & {\tt GoBack}      & quit the currently displayed item             \\
PF4     & {\tt Disp-$>$BMK} & save a bookmark for the currently             \\
        &                   & displayed item                                \\
\hline
PF5     & {\tt Curs-$>$BMK} & save a bookmark for the menu item that is in  \\
        &                   & the line where the cursor currently is        \\
PF6     & {\tt BMK}         & activate (display) bookmark file as a menu    \\
PF7     & {\tt PageUP}      & go one page back (up)                         \\
PF8     & {\tt PageDN}      & go one page forward (down)                    \\
\hline
PF9     & {\tt Save}        & save the item that is currently displayed     \\
PF10    &                   & {\it unassigned yet, reserved ***}            \\
PF11    & {\tt Display}     & display item of line where cursor is currently\\
PF12    & {\tt Circle}      & switch to next active item                    \\
\hline\hline
PF13    & {\tt First}       & goto first screen if still active             \\
PF14    & {\tt (basepfs)}   & assign PF 1-12 with base functions            \\
PF15    & {\tt Exit}        & leave Gopher immediately                      \\
PF16    &                   & {\it unassigned yet, reserved ***}            \\
\hline
PF17    & {\tt Split}       & split the screen                              \\
PF18    & {\tt Join}        & join screens                                  \\
PF19    & {\tt PageUP}      & go one page back (up)                         \\
PF20    & {\tt PageDN}      & go one page forward (down)                    \\
\hline
PF21    & {\tt ShowRing}    & show active items in XEDIT ring               \\
PF22    &                   & {\it unassigned yet, reserved ***}            \\
PF23    & {\tt Setup}       & Setup some parameters for using Gopher        \\
PF24    & {\tt Execute}     & execute command                               \\
\hline\hline
ENTER   &                   & display item of line where cursor currently is\\
\hline
\end{tabular}
\end{center}
\end{table}




PF24 triggers execution of general editor commands just like the ENTER
key would do during a normal XEDIT session while the ENTER key has the
same function as PF11, that is displaying the selected item.
Activating the {\tt Display} function will make the
Gopher client fetch the item described by the line of the menu where
the cursor is currently in, the result is then displayed as a new XEDIT
file. The file containing the menu is then still active, each fetched
item adds another file to a so-called XEDIT file ring.

Leaving a file with the {\tt GoBack} function ({\tt PF3}) causes that file
to be removed from the editor ring and XEDIT will select another file from
the ring for display.

This kind of behaviour might lead the user to a tree-like view of the
data space.  In reality --- whatever that term means --- is the user
confronted with a quite chaotic, only partially tree like data space.
One of the main features of the gopher protocol is the ability to point
to other servers.  Thus most servers point to the central node in the
world-wide gopher network, {\tt boombox.micro.umn.edu}, this server also
maintains a list of other known gopher servers.  The result is a circle
of servers pointing to each other, back and forth.



\SUBsubsection{File Ring}

The XEDIT file ring and the insertion and deletion strategy work like
Ariadne's Thread.  However, the {\tt Circle} function allows the user
to activate the next file in the editor file ring without quitting the
current file.  The function {\tt ShowRing} (PF20) gives you a list of
items in the XEDIT ring, you can select any item by placing the cursor
in the appropriate line and pressing ENTER or PF11.  All in all, the
XEDIT ring system is a perfect way to loose orientation in the
Gopher-maze...  What was it that I wanted to explain now?  Ah yes,
bookmarks!



\SUBsubsection{Bookmarks ({\tt BMK})}

After exploring Gopher Space for a while you will realize, that an
enormous quantity of information is out there.  But where exactly
did you see that particular piece of invaluable information?  You could
write down it's `coordinates'.  By the way, coordinates in Gopher Space
are three components that are found in a normal Gopher directory or menu:
server name, port number and selector string.  The item type is useful
for your client to know how a particular item should be presented, the
fifth component of a directory is only of interest for the user, it is
the so-called `display string' or name and should give you a clue about
what you can expect from a particular item.

VieGOPHER, like most other Gopher clients, allows you to take notes about
the coordinates of an item you found.  This is usually called a bookmark,
because it leads you back to the place where you found the information
and is not a reproduction of the information itself.
You can either take a bookmark for the item that is currently displayed
(PF4) or you could request that the item which is in the menu line of the
cursor should be transfered to the bookmark list (PF5).  The list of your
bookmarks is stored in a file named {\tt GOPHER.BOOKMARK} and can be
activated by pressing PF6 (the BMK function).  The bookmark file is then
read and the file is added to the XEDIT file ring.

\noindent Notes:
\begin{itemize}
\item   Save the bookmark file after modifying (editing) it using
        the FILE command.
\item   Be careful about activating and editing the bookmark file
        and taking more bookmarks while the bookmark file is still
        active in the XEDIT file ring.
\item   Save bookmarks for WAIS search items from the menu.
\end{itemize}



\SUBsubsection{Saving items to a file}

Text items can be stored in local files for later processing or printing.
This is done by pressing PF9 (save) when displaying the text item
of interest.  The text will be appended to the file {\tt GOPHER.SAVEFILE},
together with a short header listing the gopher coordinates of the item.
  Of course, the XEDIT commands {\tt SAVE}, {\tt FILE},
and {\tt PUT} can be entered on the command line as normal.



\SUBsubsection{Telnet}

VieGOPHER supports the telnet types 8 for `normal' telnet and T for
Telnet 3270.  IBM's {\tt TELNET.MODULE} is able to handle both types,
so the client will call this program by default.

It may be desireable,
however, to configure another program to handle type 8 or type T
sessions.  This can be done either at configuration time by specifying
appropriate values for the ROSE-variables {\tt CL\_TELNET\_8} and
{\tt CL\_TELNET\_T}.

The user can overide this selections by assigning the programs he
desires to the global variables {\tt TELNET-8} and {\tt TELNET-T},
both are in the group {\tt GOPHER}.

Please note, the ROSE-variables and the global variables must specify
the complete name of the desired program.  For example, a user would
use VieGOPHER's SETUP function or
enter the following command to setup Arty Ecock's VT100 telnet emulator
for `normal' telnet sessions:

\begin{verbatim}
GLOBALV SELECT GOPHER SETLP TELNET-8 TNVT100 EXEC
\end{verbatim}



\SUBsection{Supported Item Types}

Table \ref{supported.items} list all currently supported item
types.  Please be aware, some servers know item types that this
client will not be able to process, other item types (like {\tt w})
are not yet standardized and might be interpreted differently by
the server on one side and the client on the other side.



% Links: * edindex (type code overview)
\begin{table}
\caption{Supported Item Types}\label{supported.items}
\rule{0mm}{2mm}
\begin{minipage}{\textwidth}
\begin{center}
\begin{tabular}{|c|c|l|cl|}\hline
type & marker & short           &\multicolumn{2}{l|}{actions on activation}\\
\hline\hline
0    & \verb/<F>/& File            & $\circ$ & retrieve                        \\
     &        &                 & $\circ$ & display as plain text           \\
\hline
1    & \verb/<D>/& Directory       & $\circ$ & retrieve                        \\
     &        &                 & $\circ$ & display as directory            \\
\hline
2    & \verb/CSO/    & CSO Phonebook\footnote{under construction,
  limited support is provided} & $\circ$ & retrieve \\
     &        &                 & $\circ$ & display as plain text           \\
\hline
7    & \verb/<S>/& search          & $\circ$ & prompt for search string        \\
     &        &                 & $\circ$ & submit search selector to
                               server\footnote{compatible with WAIS gateway}\\
     &        &                 & $\circ$ & display result as directory     \\
\hline
w    & \verb/<S>/& search          & $\circ$ & prompt for search string        \\
     &        &                 & $\circ$ & submit search selector to server\\
     &        &                 & $\circ$ & display result as file          \\
\hline
8    & \verb/<T>/& Telnet          & $\circ$ & display selector string as a    \\
     &        &                 &         & clue for logging in             \\
     &        &                 & $\circ$ & start telnet session            \\
\hline
T    & \verb/<T>/& Telnet 3270\footnote{Type={\tt T} is now officially declared
as 3270 telnet mode.  By default, types 8 and T are handled the same way as
normal telnet, but this can be configured either at installation time or
as a user setup option.}        & $\circ$ & display selector string as a    \\
     &        &                 &         & clue for logging in             \\
     &        &                 & $\circ$ & start telnet session            \\
\hline
i    & \verb/***/    & info            & & (no action on activation) \\
%%%%%%%%%%%%%%%%%%% neither <B> nor <H> are currently used %%%%%%%%%%%%
%%% \hline
%%% B    & \verb/<B>/& BSS             & $\circ$ & retrieve                        \\
%%%     &        &                 & $\circ$ & display as
%%% file\footnote{sending the data to a file without displaying should be
%%% implemented; transparent decoding would be a useful option} \\
%%% \hline
%%% H    & \verb/<H>/& Hypertext       & $\circ$ & retrieve                        \\
%%%      &        &                 & $\circ$ & display as ordinary 
%%%                file\footnote{hypertext information is not evaluated
%%% by this client yet}    \\
\hline
\end{tabular}
\end{center}
\end{minipage}
\end{table}



\SUBsection{Global Variables}

  The client uses two different types of global variables, permanent
  ones, for storing setup parameters, in the file {\tt LASTING.GLOBALV}
  and temporary ones in memory.
 
 


\SUBsubsection{User Setup}

Table \ref{clglv} on page \pageref{clglv} lists all the
global variables that are used by the client.
They are in the group GOPHER in the file {\tt LASTING.GLOBALV}
and can be modified with the SETUP function built into the
client (PF23: {\tt Setup}) or by calling the program GOPHSTP.

%%% this part is obsolete already ... %%%%%%%%%%%%%%%%%%%%%%%%%%%%%%%%
%%%\noindent {\bf Attention:}
%%% {\tt HOST}, {\tt ITYPE} and {\tt SEL}
%%%     need to be set, accessed and modified using
%%%     {\tt GLOBALVV.EXEC} since they will usually contain mixed
%%%     case strings which can't be handled with GLOBALV.
%%%%%%%%%%%%%%%%%%%%%%%%%%%%%%%%%%%%%%%%%%%%%%%%%%%%%%%%%%%%%%%%%%%%%%%
 
\begin{table}
\caption{Permanent Global Variables for the Client}\label{clglv}
\rule{0mm}{2mm}
\begin{center}
\begin{minipage}{\textwidth}
\begin{tabular}{|l|l|}
\hline  % --------------------------------------------------------------------
% {\tt  KEEP\footnote{this feature is not fully implemented yet.}       } 
%                    & define, if items should be kept and how long         \\
% {\tt  ~~=ALL     } & keep all items                                       \\
% {\tt  ~~=SESSION } & keeps items only during a session                    \\
% {\tt  ~~={\sl{(none)}}} & don't keep any items [Default]                  \\
% \hline % -------------------------------------------------------------------
{\tt HOST}      & name of the GOPHER server to be accessed at startup       \\
                & [Default: {\tt gopher.wu-wien.ac.at}]                     \\
\hline % ---------------------------------------------------------------------
{\tt PORT}      & port of the GOPHER server                                 \\
                & [Default: {\tt 70}]                                       \\
\hline % ---------------------------------------------------------------------
{\tt ITYPE}     & item type of the first item to be fetched                 \\
                & [Default: {\tt 1} ({\sl directory})]                      \\
\hline % ---------------------------------------------------------------------
{\tt SEL}       & selector string of the first item to be fetched           \\
                & [Default: {\sl text}\footnote{set during installation
                  as a ROSE definition}]                                    \\
\hline % ---------------------------------------------------------------------
{\tt DNAME}     & name of the startup menu                                  \\
                & [Default: {\sl none}]                                     \\
\hline % ---------------------------------------------------------------------
{\tt BOOKMARK}  & name of the bookmark file                                 \\
                & [Default: {\tt GOPHER BOOKMARK A}]                        \\
\hline % ---------------------------------------------------------------------
{\tt SAVEFILE}  & name of the file where items may be saved                 \\
                & [Default: {\tt GOPHER SAVEFILE A}]                        \\
\hline % ---------------------------------------------------------------------
{\tt INFOLEVEL} & level of verbosity, amount of function key description    \\
                & [Default: {\tt 5}]                                        \\
\hline % ---------------------------------------------------------------------
{\tt TELNET-8}  & name of the telnet program for items of {\tt Type=8}      \\
                & [Default: {\tt TELNET MODULE}]                            \\
\hline % ---------------------------------------------------------------------
{\tt TELNET-T}  & name of the telnet program for items of {\tt Type=T}      \\
                & [Default: {\tt TELNET MODULE}]                            \\
\hline % ---------------------------------------------------------------------
\end{tabular}
\end{minipage}
\end{center}
\end{table}

 


\SUBsubsection{temporary variables}
 
  Table \ref{clglv-t} on page \pageref{clglv-t} lists the client's
  temporary global variables,
  %%% they are in the group GOPHER in the special format 
  %%% file {\tt LASTING.GLOBALXV}.
  the CMS program GLOBALV keeps them in main memory.
  %%% The Gopher program will purge this file
  %%% at the end of every session.
  These variables can not be accessed directly.


\begin{table}
\caption{Temporary Global Variables for the Client}\label{clglv-t}
\rule{0mm}{2mm}
\begin{center}
\begin{minipage}{\textwidth}
\begin{tabular}{|l|l|}\hline 
{\tt  TMPCNT     } & number of the last item fetched (0..??)\\
\hline
{\tt  GHOST      } & host, port, type and selector of the last item fetched\\
\hline
{\tt  GI1:{\sl{$<$ty$>$}}  } &  directory entry of the item held in 
                                file {\tt GOPHTMP.}{\sl{$<$ty$>$}}\\
{\tt  GI2:{\sl{$<$ty$>$}}  } &  {\sl $<$ty$>$} is the file type of the
                                XEDIT file in the ring\\ \hline
{\tt  L{\sl{$<$ty$>$}}     } &  screen information for active item in
                                XEDIT ring\\ 
\hline
\end{tabular}
\end{minipage}
\end{center}
\end{table}

\rule{0mm}{5mm}

%%%\noindent {\bf Attention:}
%%% {\tt GHOST}, {\tt GI1:}{\sl{$<$ty$>$}} and
%%%     {\tt GI2:}{\sl{$<$ty$>$}}
%%%     are set, accessed and modified by the client modules
%%%     using {\tt GLOBALXV.EXEC} since they will usually contain mixed case
%%%     strings, possibly even longer than 256 bytes, which can't be handled
%%%     with GLOBALV or GLOBALVV.
%%%{\tt  L{\sl{$<$ty$>$}}} is handled using GLOBALV.



\SUBsection{Files}

  There are no files read in normal operation.  For direct access without
  TCP/IP, the local fetching option ({\tt CL\_USE\_LOCALFETCH}) must be
  turned on and all the files visible to the server must also be visible
  to the client, especially the selector file, the group file, and the
  root index.
 


\subsection{Calling {\tt GOPHFTCH.EXEC}}

The heart of the VieGOPHER client, the item fetcher {\tt GOPHFTCH.EXEC}
is a program of it's own right and can be embedded in other applications
as well.  It's calling formats are:\\
\rule{0mm}{1mm}

\begin{center}\begin{tabular}{l|l|}\cline{2-2}
1. & \verb#GOPHFTCH ofn oft ofm[(fm_opt] type host port path#     \\ \cline{2-2}
2. & \verb#GOPHFTCH ofn oft ofm[(fm_opt] FILE ifn ift ifm#        \\ \cline{2-2}
3. & \verb#GOPHFTCH ofn oft ofm[(fm_opt] PULL#                    \\ \cline{2-2}
%%%%%%%%%  & \verb#fm_opt := PUSH | QUEUE | XEDIT# \\ \cline{2-2}
\end{tabular}\end{center}

\rule{0mm}{3mm}

In the first format, all Gopher coordinates are specified on the
command line, the second format requires GOPHFTCH to read the coordinates
from the specified file and the third format tells GOPHFTCH that the
coordinates can be pulled from the stack.


\verb#ofn oft ofm# specifies the file where the fetched data should be
written to, unless {\tt fm\_opt} forces the data to be processed otherwise.

\verb#fm_opt#, the filemode options, can be either empty,
{\tt PUSH}, {\tt QUEUE} or {\tt XEDIT}.\\
\rule{0mm}{1mm}

\begin{tabular}{|l|l|}\hline
Option     & Description \\ \hline \hline
           & The empty filemode option indicates that the data should\\
           & indeed be written to the specified outout file. \\ \hline
{\tt PUSH} & indicates that the data will be pushed to the stack. \\ \hline
{\tt QUEUE}& indicates that the data will be queued on the stack. \\ \hline
{\tt XEDIT}& indicates that incoming data should immediately be written\\
           & to the file being edited with XEDIT. \\ \hline
\end{tabular}

\rule{0mm}{3mm}

\noindent
Please note:
\begin{itemize}
\item   there must be no space between the output filemode, the
        bracket and the actual filemode option!
\item   Please note also, a valid filename for writing must be
        specified in all cases!
\end{itemize}

\begin{figure}
\caption{Format of the data on the stack for options
{\tt PUSH} and {\tt QUEUE}} \label{stack.format}
\begin{center}
\fbox{\parbox{8cm}{%
{\tt TOP}\\
{\tt -}{\sl complete text line}\\
{\tt +}{\sl beginning of a long text line}\\
{\tt +}{\sl more of the long text line}\\
{\tt -}{\sl end of the long text line}\\
{\tt -}{\sl another short text line}\\
{\tt END}%
}}\end{center}
\end{figure}

\noindent
{\sl type} specifies the type of the item to be fetched and controls
the way how GOPHFTCH will process the data.\\
{\sl host} specifies the internet address of the Gopher server's host.\\
{\sl port} specifies the TCP port number where the Gopher server listens.\\
{\sl path} specifies the selector string which GOPHFTCH will send to
the server to request an item.\\
{\sl ifn ift ifm}\ specifies the input file where the necessary Gopher
coordinates can be found.

\begin{figure}
\caption{Format of the Gopher coordinates on file for the FILE-method}
\begin{center}\fbox{\parbox{8cm}{\sl type\\host\\port\\path}}\end{center}
\end{figure}

\begin{figure}
\caption{Format of the Gopher coordinates on the stack for the PULL-method}
\begin{center}\fbox{\parbox{8cm}{%
{\tt TYPE=}{\sl type}\\
{\tt HOST=}{\sl host}\\
{\tt PORT=}{\sl port}\\
{\tt PATH=}{\sl path}\\
{\tt HTAP=}{\sl path}\\
{\tt END}%
}}\end{center}
\end{figure}

Long path-strings can be segmented into a series of {\tt PATH=} or
{\tt HTAP=} records.  {\tt PATH=} segments are put at the end of
the path string, while {\tt HTAP=} segments are put at the beginning
of the path string.  The {\tt END} record must be present to tell
GOPHFTCH when to stop reading from the stack.



\section{Server}
 
% * Modules
%   * GOPHERDD.EXEC
%   * Modules from other Sources
%   Installation
%   * (1)
%   * Fig: Instsec in GOPHERDD.EXEC
%   * (2)
% * Files
%   * GOPHERD.SELECTOR
%   * GOPHERD.GROUPS
%   * GOPHERD.INDEX
%   * Format of an Index File
% * Global Variables
%   * Tbl: Global Variables


\subsection{Server Modules}

%   * GOPHERDD.EXEC
%   * Modules from other Sources


\subsubsection{Multi-Threaded Multi-Protocol Daemon}
 
The file {\tt GOPHERDD.EXEC} resembles the main module of the server.
This program will also call the modules {\tt FINGERXX.EXEC} to generate
the response for finger queries and {\tt COOKIE.EXEC} to generate
short text messages from a cookie base.
 
 


\subsubsection{Modules from other Sources}

For managing TCP connections, the file {\tt REXTCPIP.MODULE} is used,
trapping control messages from servicing VMs is implemented using
{\tt YWAKEUP.MODULE} or IBM's standard WAKEUP method.
 
TCP/IP must be available to the server, however, access to the TCPIP
public disk is not necessary.
 
 


\subsection{Installation}
 
  The files {\tt GOPHERD.EXEC}\footnote{This program is now obsolete;
  it is still distributed in the package to provide the interested
  reader with some in-depth information.} as well as {\tt GOPHERDD.EXEC}
  have an installation
  section where local parameters should be configured at installation
  time using ROSE.
  There are the definitions of the ports where the servers listen
  and the files that are used by the server modules.
  Figure \ref{conf.gopherdd} shows an example.
 


\begin{figure}
\caption{ROSE Installation section in
{\tt GOPHERDD.EXEC}}\label{conf.gopherdd}
\rule{0mm}{2mm}
\begin{center}
\begin{small}
\begin{verbatim}
/* to be configured at installation time by ROSE:  - - - - - - - - - - */
if logfile=''      then logfile=      #<LOGFILE>#;
if selectorfile='' then selectorfile= #<SELECTORFILE>#;
if groupfile=''    then groupfile=    #<GROUPFILE>#;
if rootindex=''    then rootindex=    #<ROOTINDEX>#;
if gopherport=''   then gopherport=   #<GOPHER_PORT>#;
if fingerport=''   then fingerport=   #<FINGER_PORT>#;
if cookieport=''   then cookieport=   #<COOKIE_PORT>#;
if sreport=''      then sreport=      #<SRE_PORT>#;
/* - - - - - - - - - - - - - - - - - - -  end of configuration section */
\end{verbatim}
\end{small}
\end{center}
\rule{0mm}{2mm}
\end{figure}



\subsubsection{Virtual Machines}
  The setup used at {\sf awiwuw11.wu-wien.ac.at} is this:  
  the virtual machine {\bf GOPHER} has all the programs on it's minidisk 191,
  accessed as A.  This disk is public to all users.
  Data files are stored on various disks.
  The VM also has a link to GOPHERD.191 as I.

  The VM {\bf GOPHERD} has a link to GOPHER.191 as minidisk B (read-only)
  and all the necessary data disks of GOPHER and other VMs.
  The server module GOPHERDD is started from the profile\footnote{A sample
  profile is shown in section \ref{prof.GOPHERD}.}.  GOPHERD can be forced
  and auto-logged by the VM TCPIP and is currently also forced and
  auto-logged by the VM VMUTIL four times a day.

  Allowing TCPIP to control GOPHERD requires an entry in the file
  PROFILE TCPIP.  Figure \ref{PROFILE.TCPIP} on page \pageref{PROFILE.TCPIP}
  shows an excerpt of that file, listing relevant entries.
  Please note that GOPHERD is mentioned in the AUTOLOG section once and
  three times in the PORT section, where the statements for port
  17 (Quote) and 79 (Finger) are flagged with NOAUTOLOG.  It's not
  clear if it could cause problems when one virtual machine
  has more than one autolog-controlled ports and sofar this setup
  did not reveal any shortcoming.

\begin{figure}\label{PROFILE.TCPIP}
\caption{Excerpt from PROFILE TCPIP}
...\\
...
\begin{verbatim}
AUTOLOG
  FTPSERVE  PASSWD     ; FTP Server
  GOPHERD   PASSWD     ; Gopher, Finger and Quote Server
  LPSERVE   PASSWD     ; LP Server
\end{verbatim}
...\\
...
\begin{verbatim}
ENDAUTOLOG

PORT
  17 TCP GOPHERD   NOAUTOLOG ; Quote Server
  20 TCP FTPSERVE  NOAUTOLOG ; FTP Server
  21 TCP FTPSERVE            ; FTP Server
  23 TCP INTCLIEN            ; TELNET Server
  25 TCP SMTP                ; SMTP Server
  53 TCP NAMESRV             ; Domain Name Server
  53 UDP NAMESRV             ; Domain Name Server
  70 TCP GOPHERD             ; Gopher Server
  79 TCP GOPHERD   NOAUTOLOG ; Finger Server
 109 TCP POP2                ; Post Office Protocol Version 2
\end{verbatim}
...\\
...
\end{figure}



\subsection{Files}

%   * GOPHERD.GROUPS
%   * GOPHERD.DSKGRP
%   * GOPHERD.SELECTOR
%   * GOPHERD.EAI
%   * GOPHERD.INDEX
%   * Format of an Index File

The server processes a number of files to obtain vital information
for it's operation.  Purpose and format of these files is described
in the following sections.

\begin{tabular}{|l|l|} \hline
\tt GOPHERD.GROUPS    & privileged IP addresses, access groups \\ \hline
\tt GOPHERD.DSKGRP    & list of public data disks \\ \hline
\tt GOPHERD.SELECTOR  & known selector strings \\ \hline
\tt GOPHERD.EAI       & embedded application interface \\ \hline
\tt GOPHERD.INDEX     & main (or root) index \\ \hline
\end{tabular}



\subsubsection{{\tt GOPHERD.GROUPS}}\label{groupfile}
 
  This file defines the known access group names and lists the
  privileged IP addresses that are contained in a group.
  Records have the form...\\
\rule{0mm}{1mm}

\begin{center} 
    \fbox{\sl group ({\tt\symbol{32}}addr)+}
\end{center}

\rule{0mm}{3mm}

\noindent Example 1:\\ \rule{0mm}{1mm}

\begin{center}
%\fbox{\parbox{8cm}{
\begin{verbatim}
* File GOPHERD GROUPS
*
* list of know groups
gg       137.208.
wu-wien  137.208.
edvz     137.208.16.
schnurli 137.208.16.39
\end{verbatim}
%}}
\end{center}

\rule{0mm}{5mm}

\noindent where $\ldots$\\ \rule{0mm}{1mm}

\noindent\begin{tabular}{|lcl|}\hline
137.208.       &$\ldots$& is the network address of the Univ. of Economics. \\
137.208.16.    &$\ldots$& is the sub network of the computer center.        \\
137.208.16.10  &$\ldots$& is the author's PC                                \\
\hline\end{tabular}

\rule{0mm}{5mm}

\noindent NOTES:
\begin{itemize}
\item   The group {\tt *} is not defined in {\tt GOPHERD.GROUPS}.
        Files refering to group names will use {\tt *} to indicate that
        something is available to everyone on the Internet.
\item   There is a dot ({\tt .}) after subnet addresses and
        that there is no dot after a completely specified host address.
\end{itemize}



\subsubsection{{\tt GOPHERD.DSKGRP}}\label{dskgrpfile}

The disk groups file associates file modes (minidisks) with access groups,
as defined in the previously.  This allows the server to quickly decide
if files with a certain file mode can be delivered to a client,
depending on the client's Internet address.

\noindent Format of a disk group file\footnote{For an example
see appendix \ref{example GOPHERD.DSKGRP}.}:\\
\begin{center}
\fbox{{\sl disk priority groupnames}}
\end{center}
An entry line starting with {\tt *}\ or {\tt \#}\ is comment
and ignored.

\noindent {\sl disk} is the file mode for the particular disk.
Please note, {\tt X1} is not treated the same as {\tt X2} etc.

\noindent {\sl priority} when scanning the entries during a run-down,
the file with the highest priority is delivered.  Only files of
priorities larger than 0 are sent out at all.

\noindent {\sl groupnames} is a list of group names defined in the group
file (see \ref{groupfile}).



\subsubsection{{\tt GOPHERD.SELECTOR}}\label{selectorfile}

  This file lists all known selector strings (path names)
  together with associated access group names,
  owner authentication codes, and finally the file name.
  The server checks the requested item in the selector list
  and sends out the contents of the
  associated file if the file has no access restrictions.
  Records have the form...\\
\rule{0mm}{1mm}

\begin{center}
   \fbox{\sl fn.ft.fm ~~~ owner ~~~ group ~~~ selector}
\end{center}

\rule{0mm}{3mm}

\noindent where\\ \rule{0mm}{2mm}

\noindent\begin{tabular}{|l|l|}\hline
  fn          & file name\\
  ft          & file type\\
              & * indicates that the file type is taken from the\\
              & selector string.  This is used to reduce the number\\
              & of entries in the selector file and to have less\\
              & administrative overhead when installing new files.\\
  fm          & file mode\\
  owner       & a short string identifying the owner of the item\\
              & * for the general public\\
  group       & a group name defined in GOPHERD.GROUPS or\\
              & * for the general public group\\
  selector    & any text string used as selector\\ \hline
\end{tabular}

\rule{0mm}{5mm}
 
\noindent  NOTES:
\begin{itemize}
\item   Don't use characters in the selector string which are difficult
        to be processed in the command line.
\item   Group names, owner names as well as selector strings are case
        sensitive.
\item   Owner names have currently no real meaning and are ignored
\item   Note the dotted notation for file names.
\item   Include the file mode in the filename field.
\end{itemize}
 


\subsubsection{{\tt GOPHERD.EAI}}\label{eaifile}

(info is coming up later, remind me... )

Format of the path string (selector):\\
{\tt EAI-}{\sl handle}\fbox{{\tt TAB}}{\sl key-term}

The file \fbox{{\tt GOPHERD.EAI}} defines a relationship between
a selector handle, a privilege group declaration and a method
to call the embedded application.  The format of such a file is:

  \fbox{{\sl handle}\rule{3mm}{0mm}{\sl reserved}\rule{3mm}{0mm}
        {\sl group}\rule{3mm}{0mm}{\sl calling sequence}}

The calling sequence resembles a command line for calling the
embedded application.  It can contain templates that will be
replaced with specific parameters by the gopher server before
executing the command line.  These are currently defined templates:

\noindent\begin{center}
\begin{tabular}{|l|l|}\hline
template        & replacement                             \\ \hline\hline
{\tt @kw}       & key term as delivered in the selector         \\ \hline
{\tt @fn}       & filename for the output file                  \\ \hline
{\tt @ft}       & filetype for the output file                  \\ \hline
{\tt @fm}       & filemode for the output file                  \\ \hline
\end{tabular}
\end{center}




\subsubsection{{\tt GOPHERD.INDEX}}\label{rootindex}
 
  This file is the main index (root index), it is sent out whenever
  an empty selector string comes in.
 
 


\subsubsection{Format of an Index File}

  The server does not distinguish between index files and any other
  text file, therefore, index files are implicitly just text files that
  have the directory structure as defined in \cite{gophprot}.
  Thus, all records in an index file have the form...\\
\rule{0mm}{1mm}

  \fbox{\sl type display {\tt TAB} selector {\tt TAB} host {\tt TAB} port}
 
\rule{0mm}{3mm}

\noindent
\begin{tabular}{lcl}
type    & $\ldots$ & document item type (see table
                     \ref{supported.items} on page
                     \pageref{supported.items} for list)          \\
display & $\ldots$ & string displayed in the index                \\
selector& $\ldots$ & string sent to server by the client          \\
        &          & to retrieve the document\\
host    & $\ldots$ & host address of the server                   \\
port    & $\ldots$ & port number of the server                    \\
\end{tabular}

\rule{0mm}{3mm}

\noindent NOTES:
\begin{itemize}
\item  The blanks in the line above are only used for readability,
       they will not be found in an index file.
\item  The {\tt TAB} character has hex code {\tt `05'x}
       in EBCDIC ({\tt 0x09} in ASCII).
\item  The server sends an index file (like any other file) without any
       filtering.
\end{itemize}
 


\subsection{Global Variables}
 
  Global variables are used to define a couple of parameters for the
  server, they are set and accessed using GLOBALV and are found in the
  group GOPHER in {\tt LASTING.GLOBALV}.  Table \ref{srvglv} on page
  \pageref{srvglv} lists all relevant global variables for the
  multi-threaded server daemon.



\subsection{Hooks in the Server Code}

The GOPHERDD server code provides a number of hooks.  A server
administrator can install programs that are called by the server
under certain conditions.

\subsubsection{GSH\_REAC EXEC}
Executed after an REACCESS event, triggered either by a
SYS-REACCESS request from the gopher server or by a SYS-REACCESS
message.

\subsubsection{GSH\_WAIT EXEC}
Execute some commands after n seconds of idle waiting.



\begin{table}
\caption{Global Variables for the Server}\label{srvglv}
\rule{0mm}{2mm}
\begin{center}
\begin{minipage}{\textwidth}
\begin{tabular}{|l|l|}\hline
{\tt LOGFILE}       &   defines the name of the logfile file\\
                    &   (default: {\tt GOPHERD.LOG})\\  \hline
{\tt ROOTINDEX}     &   defines the name of the file with the \\
                    &   main menu (also called root index)\\
                    &   (see section \ref{rootindex}
                        on page \pageref{rootindex}) \\
                    &   (default: {\tt GOPHERD.INDEX})\\  \hline
 
{\tt SELECTORFILE}  &   defines the name of the selector file\\
                    &   (see section \ref{selectorfile})
                        on page \pageref{selectorfile}) \\
                    &   (default: {\tt GOPHERD.SELECTOR})\\  \hline
 
{\tt GROUPFILE}     &   defines the name of the group definition file\\
                    &   (see section \ref{groupfile}
                        on page \pageref{groupfile}) \\
                    &   (default: {\tt GOPHERD.GROUPS})\\  
\hline
{\tt EAIFILE}       &   defines the name of the embedded application    \\
                    &   handle file (see section \ref{eaifile}          \\
                    &   (default: {\tt GOPHERD.EAI})                    \\
\hline
{\tt DSKGRPFILE}    &   defines the name of the disk group definition   \\
                    &   file (see section \ref{dskgrpfile}              \\
                    &   (default: {\tt GOPHERD.DSKGRP})                 \\
\hline

{\tt GOPHERPORT}& defines the TCP/IP port the daemon listens\\
                & to perform 
GOPHER protocol\footnote{see \cite{gophprot}} connections\\
                & (default: port 70) \\ \hline
 
{\tt FINGERPORT}& defines the TCP/IP port the daemon listens\\
                & to perform 
FINGER protocol\footnote{see \cite{rfc1288}} connections\\
                & (default: port 79) \\ \hline
 
{\tt COOKIEPORT}& defines the TCP/IP port the daemon listens\\
                & to perform 
Quote-of-the-Day protocol\footnote{see \cite{rfc865}} connections\\
                & (default: port 17) \\ \hline
 
{\tt SREPORT}  & defines the TCP/IP port the daemon listens\\
                & to perform 
SRE protocol\footnote{no documentation on SRE is currently 
available} connections\\
                & (default: port 150)\\ \hline
\end{tabular}
\end{minipage}
\end{center}
\end{table}



\subsection{The Gopher Server in Action}

\begin{itemize}
\item   interpreting the selector string
        \begin{itemize}
        \item   {\tt FILE-*}
        \item   {\tt SPEC-*}
        \item   {\tt SYS-*}
        \item   {\tt EA-*}
        \item   selector evaluation
        \item   :from:to notation
        \end{itemize}
\item   LOG-file (+ example)
\end{itemize}




% -----------------------------------------------------------------------
\section{Server Administration Support Programs}

% * Index Editor
% * User Posting



\subsection{Index Editor}

The Index Editor provides an interactive tool for modifying index files.
The program is, as usual, set up as a XEDIT macro, the function keys are
defined in table \ref{EDINDEX} on page \pageref{EDINDEX}.
It consists of the following modules:
\begin{itemize} 
\item  {\tt EDINDEX.EXEC}:  the editor for INDEX files\\
\item  {\tt EDINDEX.XEDIT}: screen mask for INDEX files
\end{itemize}

% -----------
\begin{table}
\caption{Programmable Function Keys in {\tt EDINDEX}}\label{EDINDEX}
\rule{0mm}{2mm}
\begin{center}
\noindent\begin{tabular}{|l|l|l|}
\hline
Key     & Label             & Description\\
\hline\hline
PF1     &                   &                                               \\
PF2     & {\tt append}      & append a new record and copy data             \\
PF3     & {\tt quit}        & quit the editor indexer                       \\
PF4     & {\tt clear}       & erase information form current record         \\
\hline
PF5     & {\tt insert}      & insert a new record and copy data             \\
PF6     & {\tt delete}      & delete a record                               \\
PF7     & {\tt back}        & display previous record                       \\
PF8     & {\tt next}        & display next record                           \\
\hline
PF9     &                   &                                               \\
PF10    &                   &                                               \\
PF11    &                   &                                               \\
PF12    &                   &                                               \\
\hline
\end{tabular}
\end{center}
\end{table}



\subsection{Server Scanner}

The file {\tt GOPHSCAN.EXEC} is a program that can scan a specified server
for all it's menus.  It can be used for testing the server, or to get a
master index for that server.  The program can also scan several servers
in one session, for instance all servers on campus to get a master index
for the complete campus.

{\tt GOPHSCAN} will only scan those servers, that are specified on
invocation, so it will {\bf not} wade through all gopher space the minute
it is started.  However, even small servers turn out to have a lot of data,
so it might be possible, that one run leads to an enormous amount of data.

The program was invented for providing a seperate program with data for an
experimental `on-campus Veronica'.  The other program is not finished yet
and will be released as a seperate package, eventually.




\section{Spin-Off Programs}

% * Remote User Information Protocol: (FINGER)
%   * FINGERXX.EXEC
%   * FINGER.EXEC
% * COOKIE


\subsection{Remote User Information Protocol: (FINGER)}

%   * FINGERXX.EXEC
%   * FINGER.EXEC



\subsubsection{{\tt FINGERXX.EXEC}}
 
  This program prepares the data for a user information query and writes
  it either to a file or the console.  The program is normally either
  called by {\tt GOPHERDD.EXEC} to process a FINGER query or 
  a {\tt SPEC-FINGER} request
  from the Gopher server.  {\tt FINGERXX.EXEC} is usually not
  invoked by the user directly.  The implementation
  of the output preparation module tried to fulfill all
  requirements of RFC1288 \cite{rfc1288}
  and it can now be said that it is compliant to this standard.
 
\noindent User information has several options and consists of several parts:

\begin{enumerate} 
\item empty query string (\{C\} query)\\
     The program returns the list of all logged in virtual machines.
     RFC1288 specifies that a finger server ``MUST at least provide the
     full name'' of all online users.  The full name of the owner of a
     virtual machine is looked up in the file
     {\tt LOCAL.NAMES} using NAMEFIND.
 
\item in-depth status of a specified user (\{U\}\{C\} query)\\
     The program tries to return following information components:

\begin{enumerate}
\item user log-on status:\\
        RFC1288 requires: ``An answer MUST include at least the full name
        of the user.''  The program returns the userid, full name,
        terminal number and the phone number (local extension), provided
        that this information can be found in the lookup files.
 
\item a user information file: {\tt FINGERID.}{\sl{userid}}\\
        The contents of this file is returned
        without filtering, it will reside on one of the Gopher service
        machine's public disks so that the GOPHERDD daemon as well as any
        user invoking FINGER has read access to the file.
 
\item information from a phone book file: {\tt PHONEBK.NAMES}
\end{enumerate}

\item phone book lookup: a string containing blanks\\
     This query type is interpreted as a full name (which is invalid as
     a userid under VM/CMS) that has to be looked up in the phone book.
     Any phone book entry matching the query string will be returned.
 
\item phone number query: {\tt \#}{\sl{number}}\\
     This query type is interpreted as a phone number that has to be
     looked up in the phone book.  Any phone book entry matching the
     query string will be returned.
\end{enumerate}

  Full names are contained in the file {\tt LOCAL.NAMES}, it is looked up
  using NAMEFIND.\\
\rule{0mm}{2mm}

\begin{tabular}{|l|l|}\hline
\multicolumn{2}{|l|}{The following fields are required to be present}\\ \hline
  {\tt :nick. }    &   the userid of the virtual machine\\
  {\tt :name. }    &   the name of the virtual machine's owner\\
\hline
\end{tabular}

\rule{0mm}{5mm}

  The phone book is a file named {\tt PHONEBK.NAMES} which is processed using
  NAMEFIND.\\
\rule{0mm}{2mm}

\begin{tabular}{|l|l|}\hline
\multicolumn{2}{|l|}{The following fields are required to be present}\\ \hline
  {\tt :nick. }     & any unambigous string\\
  {\tt :name. }     & the user's full name\\
  {\tt :cnt.  }     & contents: list of all other field tags\\
\hline\hline
\multicolumn{2}{|l|}{The following fields are present at my installation}\\
\hline
  {\tt :phone. }    & the user's phone number\\
  {\tt :fname. }    & the user's first name\\
\hline
\end{tabular}

\rule{0mm}{5mm}

\noindent Notes:
\begin{itemize}
\item  A {\tt /W} token as defined in RFC1288 is
     ignored and a message indicating
     the absence of verbose information is appended to the output.
\item  Finger forwarding is actively refused by returning the message:
     ``Finger forwarding service is denied'' as defined in RFC1288.
\end{itemize}
 


\subsubsection{{\tt FINGER.EXEC}}
 
  This program is a spin-off product of the FINGER protocol
  implementation in GOPHERDD and interfaces {\tt FINGERXX.EXEC} for local and
  remote user information queries.  The program can contact finger
  servers on remote hosts and display the resulting information on the
  console.
 
 


\subsection{Quote-of-the-Day Protocol: (COOKIE or FORTUNE)}
 
  The program {\tt COOKIE.EXEC} is called by {\tt GOPHERDD.EXEC}
  to produce a short text file used as a fortune cookie served by the
  Quote-of-the-Day protocol \cite{rfc865} server.
  The program can also be invoked by the user directly and displays the
  message at the user's console.
 
  Two files are used, {\tt FORTUNE.TXT} which holds the fortune texts and
  {\tt FORTUNE.IDX} which lists the beginning and end record numbers of
  all messages in the cookie database.  Any user invoking COOKIE needs
  read access to these files, consequently, these files will be held
  on the Gopher service machine's public disk.
 
\noindent A couple of modifications seem to be appropriate:
\begin{itemize}
\item different cookie bases with different type of contents might be a
     nice idea.
\end{itemize}

\noindent Notes:
\begin{itemize}
\item  The server is RFC865 compliant but listens only on TCP port 17
     and not on a corresponding UDP port.
\end{itemize}
 
 



% -----------------------------------------------------------------------
\section{Known and Exterminated Bugs}\label{bugs}

% * REXTCPIP.MODULE: TCPopen() timing
%   * Fig: REXTCPIP.ASSEMBLE Fix 1
% * REXTCPIP.MODULE: TCPclose() hanging
%   * Fig: REXTCPIP.ASSEMBLE Fix 2
% * REXTCPIP.MODULE: GetIPAddr() name resolution
%   * Fig: TCPIP.DATA Fix 3
% * viegophr.tex: loop


\subsection{{\tt REXTCPIP.MODULE}: {\tt TCPopen()} timing}

    For some reason, the VM client could not talk to the VM server, if
    both were on the same machine.  This was a problem intrinsic to the
    original version of REXTCPIP, it is discussed here and a fix is
    presented in figure \ref{fix4} later on.

    What happened exactly, when the orginal versions were used:
\begin{itemize}
\item When the client opens a TCP connection to the server, the server
      changes its status from {\sl listening} at the port to 
      {\sl receiving},
      the connection status is then `{\tt Connected}' (Just as wanted).
\item However, the client's TCPOPEN call doesn't return so the
      client program hangs and the server's connection status is
      never changing.
      The single-threaded server could not handle other requests
      during that time, for the multi-threaded server, however,
      this wasn't really a problem, just a bit more of overhead.
\item If the client is finally killed and the TCB (TCP control block)
      gets removed
      by the system (this can be triggerd by calling NETSTAT on the
      client's VM) the TCP connection on the server side breaks down
      and the single-threaded server returns to the listen status and
      behaves normally again, the multi-threaded server simply eliminates
      that particular connection.
\end{itemize}
    One solution for
    this problem was to allow the local client to access the
    servers mini disk(s) directly.  Consequently, local connections
    with direct access to the server's files
    did not use TCPIP at all.  I was a bit lazy implementing all
    system and special functions of the GOPHERDD-server in the client
    so some of these will not work.

This kind of behaviour was finally identfied as a waiting problem 
intrinsic to
    {\tt REXTCPIP.MODULE}.  The function {\tt TCPopen()} waits for a
    {\em change} of the TCPIP machines control flag for the TCP connection.
    The flag indicates the presence of an established connection.
    The code assumes that a connection is not established without any
    noticable delay, but, in fact, the connection is established almost
    immediately and the control flag does not change anymore.  Due to that,
    this wait call will result in a loop waiting endlessly.
    Figure \ref{fix4} describes a fix for this problem where the wait
    macro is skipped for local connections.  As of now this worked fine.



\begin{figure}
\caption{{\tt REXTCPIP.ASSEMBLE} Fix 4}\label{fix4}
\rule{0mm}{2mm}
\begin{center}
\begin{small}
\begin{verbatim}
*
* Wait for the connect notificatione
*
WOPENTOP EQU  *
         OI   COMMFLAG,WAITFLAG+CONSTFLG Get new wait flag
         CLC  BLSOCKAD,BFSOCKAD
         BE   GGA
         WAIT PSW=WAITPSW,OLDPSW=WOPSW Wait on it
*        BE GGA
* GOODWILL EQU  *
*
* Got a response ... check it
*
GGA      EQU  *
         CLC  BCONN,CONNNUM            Our connection number?
         BNE  WOPENTOP                 If not, continue on wait
         L    R3,BVADB                 Get new state
*
* Return a code based on connection status.
*
         C    R3,=A(TRYOPEN)           Is connection still watiting?
         BE   WOPENTOP                 Go wait some more
*
         C    R3,=A(NOEXIST)           Connection exist still?
         BNE  GOODCONN                 If so, we got a good connetion
\end{verbatim}
\end{small}
\end{center}
\rule{0mm}{2mm}
\end{figure}



\subsection{{\tt REXTCPIP.MODULE}: {\tt TCPclose()} hanging}

% (Server hangs when connection is not closed propperly by other side)
% (fix see figure \ref{fix2})

{\tt REXTCPIP} used to wait for the other side to {\em propperly} close the
connection before the program closed the connection and control data
structures itself.  In some case this caused the server to hang because
the client wasn't either able to close the connection or didn't implment
the protocol carefully enough.  A fix in {\tt REXTCPIP.ASSEMBLE} as shown
in figure \ref{fix2} allows the server to close the connection without
waiting for the client.



\begin{figure}
\caption{{\tt REXTCPIP.ASSEMBLE} Fix 2}\label{fix2}
\rule{0mm}{2mm}
\begin{center}
\begin{small}
\begin{verbatim}
*
* Wait for connection change record
*
TOPCWAIT EQU  *
         OI   COMMFLAG,WAITFLAG+CONSTFLG   Wait for conn stat update
         WAIT PSW=WAITPSW,OLDPSW=WOPSW Wait for event
*
* Return the code given in RETCODE, and get rid of connection record
*
CLOSERET EQU  *
*        CLI  DCONNSTA,CONNCLOS        Is the connection closed?
*        BE   FINISHCL                 If so, then leave
*        CLI  DCONNSTA,NOEXIST         Is the connection gone?
*        BNE  TOPCWAIT                 if so, then leave
* Waiting is nice, but it caused more troubles than it is worth (GG)
FINISHCL EQU  *
         L    R8,NEXTCONR              Get next connection record
         ST   R8,0(R10)                Remove that link
         L    R2,DLBUFANC              Get pointer to rec buffer
         DMSFRET DWORDS=MAXBYTES/8,LOC=0(R2)  Release rec buffer
         DMSFRET DWORDS=CONRLEND,LOC=0(R9)  Drop storage
\end{verbatim}
\end{small}
\end{center}
\rule{0mm}{2mm}
\end{figure}



\subsection{{\tt REXTCPIP.MODULE}: {\tt GetIPAddr()} name resolution}
\label{bug.fdns}
 
The function {\tt GetIPAddr()} in {\tt REXTCPIP.MODULE}
ands over most of the work to the nameserver and returns whatever
he first nameserver comes up with.

The NAMESRV virtual machine might not deliver the correct IP-address as
expected, sometimes the lookup would simply fail and no IP-address
would be returned at all.
Apparently, the function {\tt GetIPAddr()} together with NAMESRV are
quite `case sensitive' so that the function returns different
IP-addresses for various upper/lower case combinations of a
given host name.

\noindent  Examples:\\
\begin{itemize}
\item (done second week of March 1992)\\
\begin{tabular}{|lcl|}\hline
  {\tt   ganymed.wu-wien.ac.at } & $\Longrightarrow$ & failed \\
  {\tt   GANYMED.WU-Wien.AC.AT } & $\Longrightarrow$ & 137.208.5.2 \\
  {\tt   GANYMED.WU-WIEN.AC.AT } & $\Longrightarrow$ & failed \\
  {\tt   Ganymed.WU-Wien.AC.AT } & $\Longrightarrow$ & failed \\
  {\tt   uicvm.uic.edu         } & $\Longrightarrow$ & 128.248.2.50 \\
  {\tt   uicvm.UIC.EDU         } & $\Longrightarrow$ & 128.248.???.??? \\
\hline\end{tabular}

\item (done third week of March 1992)\\
\begin{tabular}{|lcl|}\hline
  {\tt   ganymed.wu-wien.ac.at } & $\Longrightarrow$ & 137.208.5.2 \\
  {\tt   GANYMED.WU-Wien.AC.AT } & $\Longrightarrow$ & failed \\
  {\tt   GANYMED.WU-WIEN.AC.AT } & $\Longrightarrow$ & failed \\
  {\tt   Ganymed.WU-Wien.AC.AT } & $\Longrightarrow$ & failed \\
  {\tt   uicvm.uic.edu         } & $\Longrightarrow$ & failed \\
  {\tt   UICVM.UIC.EDU         } & $\Longrightarrow$ & 128.248.2.50 \\
\hline\end{tabular}
\end{itemize}
 
  As you can see from these examples, this situation was unsatisfying and
  so FDNS was created to provide a quick hack to
  circumvent {\tt GetIPAddr()}.
 
  After several days of discussion\footnote{with Ken Hornstein and
  Alan Flavel} and testing the following solution
  was found:  removing the entry {\tt NSINTERADDR 14.0.0.0} from the file
  {\tt TCPIP.DATA} and providing an entry pointing to a UNIX nameserver,
  as shown in figure \ref{fix3}, works reliably.
 
  Please note also that this problem can't be solved by simply
  translating the host name either to lower case or to upper case!
 
  Maybe it's possible to find even better solutions for this problem,
  if you do, please drop me a line.  In my oppinion, improving
  {\tt GetIPAddr()} is the most promising path to take.
 


\begin{figure}
\caption{{\tt TCPIP.DATA}: defining the name server}\label{fix3}
\rule{0mm}{2mm}
\begin{center}
\begin{small}
\begin{verbatim}
;
; NSINTERADDR specifies the internet address of the name server.
; LOOPBACK (14.0.0.0) is the default value (your local name server).
; If a name server will not be used, then do not code an NSINTERADDR
; statement (Comment out the NSINTERADDR line below).  This will cause
; all names to be resolved via site table lookup.
;
; NSINTERADDR  14.0.0.0
NSINTERADDR  137.208.3.4
NSINTERADDR  137.208.3.5
\end{verbatim}
\end{small}
\end{center}
\rule{0mm}{2mm}
\end{figure}



\subsection{{\tt viegophr.tex}: loop}\label{viegophr.bug}
  The list of known bugs in {\tt viegophr.tex} contains a
  non-terminating loop.
  (See \cite{viegophr.tex}, section \ref{viegophr.bug}.)

% -------------------- 


\section{(misc)}
 
% * Wishes
% * Notes
% * Acknowledgements


\subsection{Wishes}

\begin{itemize}
\item Please report any bugs you find and feel free to make
    suggestions for improvements etc...

\item Send me email if you've installed either the server or the client
    and tell me if you wish to receive updates.

%%% \item Job offers are welcome.
\end{itemize}




\subsection{Notes}
 
\begin{itemize}
\item The center of Gopher Space is {\tt gopher.micro.umn.edu}, the
      FTP archive is {\tt boombox.micro.umn.edu}.

\item There exists a different implementation for VM, sometimes refered to
      as the `Rice' VM Gopher System.  RXSOCKET and PIPES are requiered
      for this one.  For information about this system contact Rick Troth,
      his email address is {\tt TROTH@RICEVM1.bitnet}.
 
%%\item Apparently, another VM/CMS Gopher Server/Client was implemented.
%%      This one uses RXSOCKET and PIPES too.  No further details are
%%      known about it.
 
\item A Gopher server running under MVS also exists.  (mvs.udel.edu)
      The author is Steve Bacher, his email address is
      {\tt seb1525@mvs.draper.com}.
 
\item The news group {\tt comp.infosystems.gopher}
      (in former times {\tt alt.gopher})
      exists for discussion of Gopher in general,
      the Bitnet Listserv mailing list {\tt VMGOPHER@PUCC.bitnet}
      was created for discussion of Gophers running on VM.

\item Gopher has a couple of close or not so close relatives:
\begin{itemize}
\item {\bf CWIS}: Campuswide Information Systems\\
      ... some of them are Gophers in disguise, others would like to be.
\item {\bf Gopher+}:\\
      Other Gophers think he's a snob, but he is not $\ldots$\\
      ... he's just a bit more sophisticated than the ordinary
      ones!\footnote{details can be found in \cite{gophplus}.}
\item {\bf PANDA}:\\
      ... a special breeding of Gophers that is now domestified in Iowa.
      The home of panda is {\tt panda.uiowa.edu}.  Some Gophers are
      in fact cross-breedings of the archaic Gopher generation and
      Pandas (e.g. VieGOPHER).
\item {\bf WAIS}: Wide Area Information Server\\
      ... a quite powerful information retrieval package that is often
      used as the search engine of choice in Gopher Space.
      WAIS-world has it's center in
      {\tt quake.think.com} \footnote{a lot of useful information about WAIS,
      e.g. \cite{WAIS1} can be found there}.
      A very good introduction can be found in \cite{WAIS.intro}.
\item {\bf WWW}: World-Wide Web\\
      Internet-wide distributed hypertext system based on the
      HTTP protocol.  (Hypertext transport protocol)
%%%\item {\bf RCP}: Remote Cluster Protocol\\
%%%      The networked extension of Tylman Susato's NED based
%%%      open hypertext system.
\end{itemize}

\end{itemize}
 


\subsection{Acknowledgements}
 
\begin{itemize}
\item The Gopher Team of the University of Minnesota\\
      {\tt (gopher@micro.umn.edu)}\\
      The simple things prove always to be the best, great job!
 
\item Ken Hornstein\\
      {\tt (kxh105@psuvm.psu.edu)}\\
      Without his {\bf REXTCPIP} neither the client nor the server
      would exist at all.

\item Arthur Ecock (City University of New York, CUNY)\\
      {\tt (eckcu@cunyvm.cuny.edu)}\\
      Let's hope more programs are designed, implemented and
      documented in such an excellent way like {\bf RXSOCKET}.
 
% \item Jeanette L. Jones (University of Illinois, Chicago)\\
%       {\tt (u25177@uicvm.uic.edu)}\\
%       She translated my `somewhat english' version of this document into
%       comprehensive english.
 
\item Alan J. Flavell (University of Glasgow)\\
      {\tt (sy07@i1.ph.gla.ac.uk)}\\
      His suggestions were used to improve the code and to clean up
      some of the major irregularities.
 
\item Mary Posey (University of Maryland, UMCP)\\
      {\tt (mposey@umdacc.umd.edu)}\\
      She spent an enormous amount of time and energie to point out and
      remove inconsistencies in the code.

\item John Sroka (San Jose State University)\\
      {\tt (sroka@sjsuvm1.sjsu.edu)}\\
      Help?  He wrote the help text, I just took it :)).

\item ... and finally, this girl in Chicago for reminding me who the
      audience is.
\end{itemize} 



\section{Implementation Notes}
 
This section gives informal notes about VieGOPHER, e.g.
some variables and details of the servers and client modules.
Neither completeness nor usefulness for any purpose can be assumed.
 
% * GOPHERDD.EXEC
%   * variables
%   * slot types and ports
%   * status codes for a slot
%   * other variables
 


\subsection{Discussion: Why are things the way they are?}



\subsubsection{Client: writing files to disk temporarily}

I was asked more than once why the client writes files received from
the server to disk first, reads it into editor memory and then discards
the disk file.  Alan Flavel suggested that {\tt GOPHFTCH.EXEC}
should push lines to the stack, wrapped into {\sf XEDIT} red tape.
At this point, finally, this shortcoming is more or less
resolved.\footnote{If you want more backgroud information about the
way VieGOPHER operates in certain situations, please drop me a line...}

\noindent
There were several reasons why the file write method was used:\\
{\bf First:}  the client started as a single module that had to
fetch items and store them to disk.  This was initially the only
purpose of {\tt GOPHFTCH.EXEC}.  Later, as interactive XEDIT
support was added, the implementation of {\tt GOPHFTCH.EXEC}
remained more or less intact.\\
{\bf Second:}  there are in fact items in Gopher Space that have
record sizes of more than 256 bytes, e.g. directories from
WAIS-to-Gopher gateways sometimes deliver records in the range
of 600 bytes per line.  A record on the stack can only be
255 byte long, so using the stack for holding incoming data
items requires a method for breaking up long lines.  This fact
should explain the format of data on the stack described in
figure \ref{stack.format} on page \pageref{stack.format}.

\rule{0mm}{5mm}

The current implementation of VieGOPHER features another method
of bringing data to the screen.  In a typical gopher session, the
fetcher module {\tt GOPHFTCH.EXEC} is called with the {\tt XEDIT} file mode
option.  This option allows the fetcher to transfer data directly to
XEDIT data buffer and commands XEDIT to refresh the screen after the
first couple of lines in order to bring the data to screen as fast as
possible.  For long files this has the effect that a user can
read the first screen while the rest of the file is still being transfered.

Unfortunately, this version still does not feature an interrupt method
or a transparent interaction method, so the user is required to wait
until the data transfer is finally over.  During the transfer, XEDIT
displays the number of lines transfered sofar in the message area.

For transfering index files that originate from a type 7 search,
the write - read - erase method is used as described before.  The
reason for doing so originates from the fact that WAIS searches ---
these generate type 7 indexes --- can produce enourmously long
selector strings (paths).  This compromise allows to circumvent problems
with extremely long lines which otherwise makes it impossibly to
successfully process incoming data.

%{\bf Conclusion:}  Even when I know that  passing data via disk
%is not so  optimal I'll have to leave things as they are right
%now, because eliminating this behaviour would cost a considerable
%amount of energy which I think should be better spent at more
%problematic spots in or around VieGOPHER for now.
%Appendix \ref{to-do} at page \pageref{to-do} lists more things than
%I would like to find there.

%{\bf P.S.:}  If someone is able and willing to provide code
%for a better approach, it will be appreciated.




\subsection{{\tt GOPHERDD.EXEC}}
 
The multi-threaded server handles several connections by using a number
of so-called slots and a number of variables where information for each
slot is recorded.  All slots are checked and serviced periodically.  The
server starts by posting a LISTEN call for one or more different ports
and assinging the status for LISTEN to these slots.  When a connection
on a port is established the server immediately assigns a new slot for
that server type and post's the next listen call on the same port.
Periodically, the program checks which slot it has to serve and
eventually if a service is finished and the slot's status is changed
to UNASSIGNED.  If a connection crashes or behaves in some strange way
the slot status is changed to dead and is not serviced any longer.
A recovery strategy for the case of severe problems is currently in
design.


%   * variables
%   * slot types and ports
%   * status codes for a slot
%   * other variables


\subsubsection{variables describing information about a slot}
\noindent\begin{tabular}{|l|l|}\hline
   {\tt mtmax  }    & number of used slots \\
   {\tt mtact  }    & number of slot currently served \\ \hline

   {\tt mtgcn. }    & connection number \\
   {\tt mtsta. }    & slot status (see below) \\
   {\tt mtprt. }    & port number assigned for the slot \\
   {\tt mttyp. }    & type of the slot and consequently the service: \\
   {\tt mtfad. }    & foreign address of the connection \\
   {\tt mtsel. }    & selector string received for the slot \\
   {\tt mtsel1.}    & prefix selector string for EXTRA slots \\
   {\tt mtsel2.}    & postfix selector string for EXTRA slots \\
   {\tt mtmsg. }    & message to be sent \\
   {\tt mtfil. }    & file to be sent \\
   {\tt mtrec. }    & records of the file sent sofar \\
   {\tt mtsiz. }    & last number of record to be sent \\
   {\tt mtttw. }    & maximum time to wait \\
\hline\end{tabular}
 


\subsubsection{slot types and ports}
   The following slot types define the service they are providing.
   Port numbers are officially assigned ones.

\noindent\begin{tabular}{|l|r|l|}\hline
   type     & port   &  notes \\
\hline\hline
   GOPHER   & 70     &  the `ordinary' Gopher server \\
   FINGER   & 79     &  a Finger server \\
   COOKIE   & 17     &  a Quote-of-the-Day server \\
   SRE      & 150    &  (experimental) \\
\hline\end{tabular}
 


\subsubsection{status codes for a slot}
\noindent\begin{tabular}{|l|l|l|}\hline
   0    & {\tt UNASSIGNED }  & the slot is unused \\
\hline
   1    & {\tt DEAD       }  & a connection crashed on that slot \\
\hline
   1000 & {\tt LISTEN     }  & a LISTEN call is posted to the TCPIP system \\
\hline
   1001 & {\tt LISTEN} again & after Trying to open resume with \\
        &                    & post-processing of Listen \\
\hline
   1005 & {\tt Trying...}    & after Listening a connection was\\
        &                    & coming up slowly, the status changed\\
        &                    & to {\tt Trying to open} \\
\hline
   1010 & {\tt GET-SELECTOR} & a selector string is expected on the now \\
        &                    & established connection \\
\hline
   1020 & {\tt TRANSFER   }  & a file transfer is in progress \\
\hline
   1090 & {\tt CLOSING    }  & the connection is in the process of closing \\
\hline
   2000 & {\tt SRE LISTEN }  & wait for a SRE connection \\
\hline\end{tabular}




\subsubsection{other variables}
\noindent\begin{tabular}{|l|l|}\hline
  {\tt  deadcount } &  number of dead slots \\
\hline\end{tabular}



\begin{appendix}

% * Sample Installation
% * Packing List
% * files with configuration
% * Sample Applications
% * Authors Address
% * Obtaining the Package


\section{Sample Installation}

% * Virtual Machines, etc.
% * Profile of GOPHERD
% * GOPHERD.SELECTOR
% * GOPHERD.GROUPS
% * GOPHERD.INDEX


\subsection{Virtual Machines, etc.}



\subsection{Profile of the virtual machine GOPHERD}\label{prof.GOPHERD}
\begin{small}
\begin{verbatim}
/* ------------------------------------------------------------------- */
/* PROFILE EXEC for GOPHERD@AWIWUW11.bitnet                            */
/*                                                                     */
/* 1992 05 05                                                          */
/* ------------------------------------------------------------------- */
logfile='GOPHERD LOGFILE'
chief='GOPHER'
 
/* Act 1: do whatever might be useful */
'TERM HILIGHT ON'
'DTRIPF NOPAN'          /* who knows what this is good for? */
'TERM ESCAPE OFF'
 
/* Act 2: Log everything that happens */
'SPOOL CONS START'
'SPOOL CONS TO' chief
 
/* Act 3: Prepare the stage for the next show */
'SENDFILE' logfile chief
'ERASE' logfile
 
/* Act 4: link necessary disks */
parse value diagrc(8?,'LINK GOPHER 191 201 RR') with CPRc .
'ACCESS 201 B'
parse value diagrc(8?,'LINK TCPMAINT 592 197 RR') with CPRc .
'ACCESS 197 X'
 
/* Act 5: run the server now */
'GOPHERDD'              /* start the *MULTI THREADED* server now ... */
 
/* Act 6: finale? nobody knows .... */
\end{verbatim}
\end{small}



\subsection{\tt GOPHERD.SELECTOR}
\begin{small}
\begin{verbatim}
* File GOPHERD SELECTOR
*
* list of known items
*
* 1992 03 04: <Gerhard.Gonter@wu-wien.ac.at>
* -----------------------------------------------------------------------
* root directory *****
ABOUT.GOPHER.B      gg * about
LOCAL.INDEX.B       gg * local
WUGOPHER.INDEX.B    gg * wu
ATGOPHER.INDEX.B    gg * austria
OTHER.INDEX.B       gg * others
EUROPE.INDEX.B      gg * EUROPE.INDEX
VMGOPHER.INDEX.B    gg * VM Gopher
DOCUMENT.INDEX.B    gg * documents
SERVICES.INDEX.B    gg * services
POINTERS.INDEX.B    gg * pointers
NETINFO0.INDEX.B    gg * net info
BSS.INDEX.B         gg * bss stuff
PROMARE.*.B         gg * PROMARE.INDEX
WAISES-1.INDEX.B    gg * WAISes (everything)
* VM Client & Server software directory *****
VMGOPHER.STATUS.B   gg * VMGOPHER.STATUS
VMGOPHER.DOC.B      gg * VMGOPHER.DOC
VMGOPHER.IMPLEMEN.B gg * VMGOPHER.IMPLEMEN
GOPHERD.EXEC.B      gg * GOPHERD.EXEC
GOPHERDD.EXEC.B     gg * GOPHERDD.EXEC
GOPHER.EXEC.B       gg * GOPHER.EXEC
GOPHER.XEDIT.B      gg * GOPHER.XEDIT
XG.EXEC.B           gg * XG.EXEC
GOPHFTCH.EXEC.B     gg * GOPHFTCH.EXEC
GOPHXFTC.XEDIT.B    gg * GOPHXFTC.XEDIT
GOPHXBMC.XEDIT.B    gg * GOPHXBMC.XEDIT
GOPHXBMI.XEDIT.B    gg * GOPHXBMI.XEDIT
GOPHXBMS.XEDIT.B    gg * GOPHXBMS.XEDIT
GOPHXSVI.XEDIT.B    gg * GOPHXSVI.XEDIT
GOPHXQIT.XEDIT.B    gg * GOPHXQIT.XEDIT
GOPHSRCH.XEDIT.B    gg * GOPHSRCH.XEDIT
EDINDEX.EXEC.B      gg * EDINDEX.EXEC
EDINDEX.XEDIT.B     gg * EDINDEX.XEDIT
GLOBALVV.EXEC.B     gg * GLOBALVV.EXEC
GLOBALXV.EXEC.B     gg * GLOBALXV.EXEC
FDNS.EXEC.B         gg * DNS.EXEC
FDNS.IN.B           gg * FDNS.IN
VMGOPHSF.EXEC.B     gg * VMGOPHSF.EXEC
* test index: WRITE PUBLIC ***************************
TEST.INDEX.B         * * test
***AUTOMATIC***
FORTUNE.TXT.B       gg * FORTUNE.TXT
\end{verbatim}
\end{small}



\subsection{\tt GOPHERD.INDEX}
\begin{vsmtt}
\begin{verbatim*}
0About Internet Gopher%about%awiwuw11.wu-wien.ac.at%70
1Local Info%local%awiwuw11.wu-wien.ac.at%70
1Gophers around the house%wu%awiwuw11.wu-wien.ac.at%70
1Gophers around the place%austria%awiwuw11.wu-wien.ac.at%70
1Gophers around the world%others%awiwuw11.wu-wien.ac.at%70
1documents%documents%awiwuw11.wu-wien.ac.at%70
1services%services%awiwuw11.wu-wien.ac.at%70
1external services%1/.exser.ind%olymp.wu-wien.ac.at%70
1life, the NET and everything%net info%awiwuw11.wu-wien.ac.at%70
1The Guide!%1/.guide.ind%olymp.wu-wien.ac.at%70
1VieGOPHER: VM/CMS Gopher Version%VM Gopher%awiwuw11.wu-wien.ac.at%70
1Test: use at your own risk%test%awiwuw11.wu-wien.ac.at%70
1Hypertext Documents  (Items are Type=H)%1/.hyx%olymp.wu-wien.ac.at%70
\end{verbatim*}
\end{vsmtt}

\noindent Note: {\tt \%} represents the TAB character, hex code
{\tt `05'x} in EBCDIC ({\tt 0x09} in ASCII).


% -----------------------------------------------------------------------


\subsection{\tt GOPHERD.GROUPS}
\begin{small}
\begin{verbatim}
* File GOPHERD GROUPS
*
* list of know groups
*
* 1992 03 10: <Gerhard.Gonter@wu-wien.ac.at>
* -----------------------------------------------------------------------
gg       137.208.
wu-wien  137.208.
edvz     137.208.16.
schnurli 137.208.16.39
\end{verbatim}
\end{small}



\subsection{\tt GOPHERD.DSKGRP}\label{example GOPHERD.DSKGRP}
\begin{small}
\begin{verbatim}
* File GOPHERD DSKGRP
*
* list of known disk groups
*
* 1992 05 25: <Gerhard.Gonter@wu-wien.ac.at>
* -----------------------------------------------------------------------
B1      1       *
C1     10       umdacc schnurli
D2     12       wu-wien
E1      0       *
\end{verbatim}
\end{small}

\noindent Comments:
\begin{itemize}
\item B1 is a public disk.
\item C1 is available to groups {\tt umdacc} and {\tt schnurli} only.
\item D2 is available to group {\tt wu-wien} alone.
\item E1 is unavailable because it has priority 0.
\end{itemize}



\subsection{\tt GOPHERD.EAI}\label{example GOPHERD.EAI}
\begin{small}
\begin{verbatim}
* FILE GOPHERD EAI
*
* embedded application definition
*
* 1993-05-25
\end{verbatim}
\end{small}


\section{Resource Guide}



%\pagebreak[1]
\subsection{Packing List}

The core of the VieGOPHER package consists of a number of modules that
are grouped in four sub-packages.  You should receive all modules in
just one distribution file, arranged as a ROSE product file.  The table
below lists all modules, describes briefly what they do and what sub-package
they belong to.\\
\rule{0mm}{2mm}

\noindent\begin{footnotesize}
\begin{tabular}{|l||c|c|c|c|c||l|} \hline
file name               &SRV&CL&SUP&EXT&OBS&description                     \\
\hline\hline
{\tt COOKIE.EXEC      } & X &  &   & X &   &produce a cookie message        \\
{\tt EDINDEX.EXEC     } &   &  & X &   &   &edit Gopher index files         \\
{\tt EDINDEX.XEDIT    } &   &  & X &   &   &mask for editing Gopher index   \\
%%% {\tt FDNS.EXEC        } &   &X &   &   & X &`Faked' DNS resolver            \\
%%% {\tt FDNS.IN          } &   &X &   &   & X &support for FDNS                \\
{\tt FINGER.EXEC      } &   &  &   & X &   &client interface for Finger     \\
{\tt FINGERXX.EXEC    } & X &  &   & X &   &Finger response generator       \\
\hline
%%% {\tt GLOBALVV.EXEC    } &   &X & X &   & X &augmented GLOBALV               \\
%%% {\tt GLOBALXV.EXEC    } &   &X &   &   & X &augmented GLOBALV (diff. format)\\
%%% \hline
{\tt GOPHCHRL.EXEC    } &   &  & X &   &   &check READERLIST for posted items\\
{\tt GOPHINST.EXEC    } &   &  & X &   &   &postinstaller calle by Rose\\
{\tt GOPHER.EXEC      } &   &X &   &   &   &client main module              \\
{\tt GOPHER.XEDIT     } &   &X &   &   &   &setup for client main module    \\
%%% {\tt GOPHERD.EXEC     } & X &  &   &   & X &server, single-threaded         \\
{\tt GOPHERDD.EXEC    } & X &  &   &   &   &server, multi-threaded          \\
{\tt GOPHFTCH.EXEC    } &   &X &   &   &   &item fetcher                    \\
{\tt GOPHRING.XEDIT   } &   &X &   &   &   &present XEDIT ring list on screen\\
%%% {\tt GOPHRNG2.XEDIT   } &   &X &   &   &   &setup for XEDIT ring presentation\\
{\tt GOPHRNG3.XEDIT   } &   &X &   &   &   &display selected item from ring\\
{\tt GOPHSCAN.EXEC    } &   &  & X &   &   &scan a (group of) gopher server(s)\\
{\tt GOPHSCRN.XEDIT   } &   &X &   &   &   &screen splitting and joining    \\
{\tt GOPHSRCH.XEDIT   } &   &X &   &   &   &mask for entering search string \\
{\tt GOPHSTOR.EXEC    } &   &  & X &   &   &store posted items              \\
\hline
{\tt GOPHXBMC.XEDIT   } &   &X &   &   &   &save a bookmark (cursor)        \\
{\tt GOPHXBMI.XEDIT   } &   &X &   &   &   &save a bookmark (displayed item)\\
{\tt GOPHXBMS.XEDIT   } &   &X &   &   &   &show bookmarks                  \\
{\tt GOPHXFTC.XEDIT   } &   &X &   &   &   &fetch selected item             \\
{\tt GOPHXQIT.XEDIT   } &   &X &   &   &   &leave displayed item            \\
{\tt GOPHXSVI.XEDIT   } &   &X &   &   &   &save displayed item             \\
\hline
%%% {\tt VMGOPHSF.EXEC    } &   &  & X &   & X &send Gopher package (sendfile)  \\
{\tt XG.EXEC          } &   &X &   &   &   &call XEDIT, Gopher style        \\
%%% {\tt VIEGOPHR.TEX     } & X &X & X & X &   & this file                      \\
\hline
\end{tabular}
\end{footnotesize}



% -----------------------------------------------------------------------
\subsection{Obtaining the Package}\label{sources}

The VieGOPHER package can be obtained from a number of sources:
\begin{enumerate}
\item from the FTP server of the University of Minnesota (Gopher Central):\\
      {\tt boombox.micro.umn.edu:/pub/gopher/Vienna\_CMS/} or\\
      {\tt boombox.micro.umn.edu:/pub/gopher/VieGOPHER/}
\item from the Gopher server of the University of Economics in Vienna:
      \begin{verbatim}
      Name=VieGOPHER: VM/CMS version
      Path=VM Gopher
      Type=1
      Host=awiwuw11.wu-wien.ac.at
      Port=70
      \end{verbatim}
\item from the FTP server of the University of Economics in Vienna:\\
      {\tt ftp.wu-wien.ac.at:/pub/src/VieGOPHER/}
\end{enumerate}



\subsection{Obtaining Third-Party Modules}

The following table gives a very brief overview over third-party modules,
what they do and which sub-packages they are used for.  Source files and
documentation are not necessary for running the VieGOPHER system.\\
\rule{0mm}{2mm}

\noindent\begin{footnotesize}
\begin{tabular}{|l||c|c|c|c|c||l|} \hline
file name               &SRV&CL&SUP&EXT&OBS&description                     \\
\hline\hline
{\tt REXTCPIP.MODULE  } & X &X &   & X &   &TCP/IP transport package        \\
{\tt REXTCPIP.ASSEMBLE} &   &  &   &   &   &+ source                        \\
{\tt REXTCPIP.MEMO    } &   &  &   &   &   &+ description                   \\
\hline
{\tt RXSOCKET.MODULE  } &   &X &   &   &   &TCP/IP transport package        \\
{\tt REXXWAIT.MODULE  } &   &  &   &   &   &general event handler           \\
\hline
{\tt YWAKEUP.MODULE   } & X &  &   &   &   &trapping messages               \\
{\tt YWAKEUP.HELPCMS  } &   &  &   &   &   &+ description                   \\
\hline
\end{tabular}
\end{footnot