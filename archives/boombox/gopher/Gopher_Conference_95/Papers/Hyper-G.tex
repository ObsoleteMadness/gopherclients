\documentstyle[12pt]{article}

\evensidemargin0cm
\oddsidemargin0cm
\textwidth15cm
\textheight20cm

\begin{document}

\title{The Hyper-G Information System}

\author{Klaus Schmaranz\\
(Institute for Information Processing and\\
Computer Supported New Media (IICM),\\
Graz University of Technology, Austria\\
kschmar@iicm.tu-graz.ac.at)}

\maketitle


\begin{abstract}

In this paper a brief description of Hyper-G - the first second
generation hypermedia system - is given. The first section identifies
some problems with first generation hypermedia systems. The following
sections are an outline of the new concepts that are implemented in
Hyper-G. These new concepts such as a world-wide distributed network
database, a separated link database and bidirectional links allow
highly sophisticated navigation, hyperlinks in all native document
types such as hypertext, images, PostScript documents and even in
movies, sound and 3D scenes. Hyper-G also implements very powerful
search mechanisms such as boolean search on titles, keywords, even
fulltext with user defined scope from one collection on one server to
all servers worldwide. And last but not least Hyper-G is fully
compatible to first generation systems like Gopher and WWW.

\end{abstract}

\section{Introduction}

Currently, the most popular Internet information systems use
distributed menus and searching (Gopher, see [Alberti et al 1992]) or
hypertext documents (WWW, see [Berners-Lee et al 1994]) to represent
information spaces. 

In both cases information is stored in a simple structured fashion on
servers, and can be accessed via clients, with clients available for
most major hardware platforms. Both systems provide an easy to use
interface and are easy to install for service providers as well as for
readers, but both systems also have some well-known weaknesses:

\begin{itemize}

\item In both systems the knowledge of certain server addresses is
required for browsing.

\item In WWW Hyperlinks as the only structuring tool (Robert Cailliau:
``the world wide spaghetti bowl'') lead to the ``lost in hyperspace''
syndrom: after following a number of links it is nearly impossible to
find the way to an interesting page again the next time a reader
searches it. Furthermore there is no way to decide how much of the
information has already been seen because the hierarchical structure
is missing.

\item Hyperlinks are made up using URLs (Universal Resource Locators)
- they point to the location of a document. If a document moves or
disappears the hyperlinks point to nowhere. It is desirable to make up
hyperlinks using URNs (Universal Resource Names) that identify the
document itself instead of the location.

\item Both Gopher and WWW servers are difficult to administrate as the
number of documents grows because there is no underlying database.

\item Search mechanisms, depending on the search engine, are mostly
limited to a simple Title search.

\item Both Gopher and WWW are lacking user identification and billing
mechanisms. 

\end{itemize}

Hyper-G [see Maurer 1994], [Andrews et al 1994] as the first second
generation hypermedia system tries to overcome the weaknesses
mentioned above using a completely new underlying concept [see Kappe
et al 1993]. One of the major design goals of Hyper-G was it to be
fully compatible to Gopher and WWW and therefore to fit perfectly into
the existing Internet landscape. Naturally the modern design of
Hyper-G allows to add many new sophisticated navigation, editing and
authoring features as you will read in the following sections. With
some lack of the special Hyper-G functionality Hyper-G servers can
naturally be accessed with all well-known Gopher and WWW clients.

\section{The Hyper-G Database}

Hyper-G is based on a client-server architecture with servers
available for all major Unix platforms and clients available for Unix
and MS-Windows platforms. A native Macintosh client is under
development. Naturally one can also use all well-known Gopher and WWW
clients to access a Hyper-G server.

The client-server protocol of Hyper-G is a connection oriented
protocol instead of a connectionless protocol as implemented in Gopher
or WWW. This allows it to implement user access rights and billing
mechanisms in Hyper-G.

The Hyper-G server itself is based on a distributed network database
architecture with separated database engines for documents and
hyperlinks. Using this concept the whole world of Hyper-G servers
presents itself to the user as one large single database with no need
to know server addresses to access single servers. The server
structure is completely hidden to the user. 

Each document in Hyper-G has an additional Object description where
all the important Attributes of a document are stored. Attributes
include title, keywords, type, creation time, last modification time
and also some very special attributes like the expiry date of a
document if any. Expiry dates are very important for quality assurance
- who has not already seen a call for papers for a conference that was
months ago! In Hyper-G a document is no longer accessible after the
expiry date and also all the links pointing to it are invisible.

Object attributes are searchable allowing very sophisticated searches
like ``find all documents that are written by a special author and not
older than half a year'' or ``find all text or PostScript documents
that have the term multimedia in their title or keywords''.

All Objects in Hyper-G have a world-wide unique Object ID that allows
to locate them even across server boundaries.

Due to the separate link database all links in Hyper-G are
bidirectional. Visiting a document a reader can not only keep track of
all the links pointing from this document to the outside world but
also can have a look at all links pointing from outside to this
document. 

The second and much more important feature of bidirectional links is
that they allow to keep the link structure consistent! If a document
is moved from one location to another, even across server boundaries
the link is pointing to the document at its new location. Hyperlinks
in Hyper-G are based on the concept of URNs (Universal Resource Names)
instead of URLs (Universal Resource Locators).

Another very important feature of the separated link database is the
possibility to add links to all supported document types, even
hyperlinks to and from images, PostScript documents, movies, 3D scenes
and sound are supported! Since the links are stored outside the
document there is no need to change format specifications of standard
formats to make them hyperlinkable.

\section{User Accounts and Billing}

Handling user accounts in Hyper-G is done similar to the way Unix
handles them: each user can belong to one or more groups and each user
has a personal home-collection where he can store his personal
documents or references to existing documents. Access to documents is
then controlled by read, write, and unlink permissions.

Billing mechanisms in Hyper-G are at the moment implemented following
2 different paradigms: The first method is to give the user a certain
amount of ``money'' and to give charged documents a certain
price. Each time a charged document is downloaded the price of the
document is subtracted from the actual funds of the user. In this case
the user pays in advance. 

The second method is based on the library paradigm and mostly used for
electronic publishing: The library that operates the server subscribes
for a certain number of exemplars of a journal, e.g. 3
exemplars. Users have access to the journals on the library server,
depending on their access rights, in the case of libraries normally
also anonymous users have access. But for e.g. 3 exemplars one
document is only accessible by a maximum of 3 simultaneous users, and
after a user has ``borrowed'' the document it is not accessible for a
given time, e.g. 1 hour. If more than the maximum allowed number of
users wants to get access the users get a ``at the moment not
available'' message.

\section{Structuring of data}

Structuring of documents in Hyper-G is done using a collection
hierarchy very similar to Gopher's file hierarchy. Additional
structuring tools are Hyperlinks as known from WWW and a special
clustering mechanism. Documents can be members of arbitrary many
collections, even across server boundaries, in this case they are
represented as references in the collection tree. Collections can
naturally not only hold native Hyper-G objects but also references to
Gopher and WWW documents, telnet objects and many more. Also
completely exotic document types can be referenced by a special
generic object that holds the object description and additional
information, e.g. which viewer to start for this type of object.

Clusters in Hyper-G are several documents that are grouped together
for logical reasons. One example of a logical cluster would be an
image and a text describing it or a movie and additional sound to the
movie. When clicking on a cluster the reader gets all documents
contained, in case of the movie and sound example the movie player
will be started and the sound played simultaneously.

One even more important feature of clustering is the support of
multilinguality. One can store several documents in different
languages in a cluster, even togegther with other arbitrary
documents. Hyper-G users have the possibility to choose their
favourite language and in case of a multilingual cluster will get the
document in their preferred language if it exists. One could for
example store an image together with an English and a German
description in a clusters and users that have chosen English as their
preferred language would then get the image with the English
description.

\section{Caching and Replication}

Due to the exponential growth of Internet and of information stored in
Internet scalability of Hyper-G was an important design goal. During
rush hours data transfer over the Internet is unacceptably slow,
especially for overseas connections. Due to this fact caching and
mirroring mechanisms had to be implemented.

Caching in Hyper-G is implemented using well-known principles, but
there is another specialized mechanism used for mirroring data between
servers - replication:

Documents already existing on one server can be exported from the
database and mirrored to other servers. In this case they should
obtain a special replica-attribute to let the server know that they
are mirrored objects from another server. The server then evaluates
the replica-attribute for remote objects as follows:

Let us assume following scenario: A user in the USA starts his Hyper-G
client and connects to his nearest Hyper-G server. Since the Hyper-G
database seamlessly hides server boundaries this user surfs a little
bit in hyperspace until he finds an interesting document in
Austria. He decides to get this document. Normally all data would be
transferred from Austria to the USA. But this document is for example
a paper of an electronic journal and the journal is completely
mirrored to the server in the USA - only the user did not know. In
this case the Hyper-G server in the USA automatically detects that it
has stored a replica of the document of Austria and sends the user the
local replica instead of unnecessarily downloading the document from
Austria.

\section{Native Document Types}

Hyper-G supports a number of native document types as well as external
documents such as references to Gopher and WWW documents and generic
objects.

Native document types are at the moment:
\begin{itemize}

\item Images of many different formats

\item MPEG movies

\item PostScript documents

\item Sound files

\item 3D scenes.

\end{itemize}

All of the above formats are fully hyperlinkable - in the case of MPEG
movies the clickable areas of hyperlinks even can move and change
their size and shape in a hyperlinked film.

PostScript documents are often used for archive material that is not
in hypertext format or also very often for specialized text,
e.g. mathematical papers containing lots of formulae that cannot be
handled very well by simple hypertext formats. Full support of
hyperlinks in PostScript adds the functionality that one normally
knows from hypertext. To add full hypertext functionality a method to
do full text search in PostScript files is under development and will
be available soon. Also an advanced PDF viewer similar to Adobe
Acrobat is under development.

But, Hyperlinkable does not only mean that one can define clickable
areas in documents, so called source anchors. One can define
destination anchors such as a special sequence in a soundfile. One
could for example have a text document with the text of a song and if
one clicks on a certain text line the sound player jumps to that part
and plays the music that belongs to this line.

Of special interest especially for the future when 3D hardware gets
cheaper and computing power increases even for the low end computers
is the 3D support. Hyper-G at the moment supports 3D scenes in
WaveFront format naturally again with full hyperlink support.

Together with the University of Minnesota and NCSA full VRML support
is under development, the first prototype of this new 3D viewer
generation is already implemented in Harmony - the Unix Hyper-G
client.

\section{Navigation}

As has already been said Hyper-G presents itself to the user as a
single structured network database. The representation of this network
database in Hyper-G clients is normally a directory-tree like
structure of collections. In Harmony, the Unix Hyper-G client, also a
second representation of the collection hierarchy is implemented - the
representation as a 3D information landscape. Collections and
documents are represented as buildings in a landscape and the user can
fly through the landscape.

Navigation also includes very advanced features such as the local
map. Based on the bidirectional nature of links one can create a local
map on the fly showing arbitrary levels of incoming and outgoing
links. This is very interesting if one finds an interesting document;
in most cases other interesting documents on the same topic can be
found by following the links to and from this document.

Very important for navigation in Hyperspace is searching. Hyper-G
supports boolean title-, keyword- and fulltext-search with user
definable scope from one single collection on one single server to all
collections on all servers worldwide. But what if the search finds
several hundreds of documents? For this case there is a special
feature in the clients, the so called ``location feedback'' - a simple
click on a document in the search result window opens the path to the
document. So the user can decide if this document is of interest for
him. Let's take for example a simple search for the item ``grep''. The
user wanted to know the exact syntax and therefore searches for a
manpage on that topic. The search could result in 2 search hits, one
in the hacker's jargon and the other one in the manpages. A single
click on the first object immediately shows that this hit is in the
collection ``hacker's jargon'' and is therefore not interesting for
the user without forcing the user to download the document first.

\section{Editing and Authoring with Hyper-G}

As already mentioned Hyper-G is not a read-only system where users can
easily view documents, Hyper-G databases allow read and write access
according to the user access rights. Documents can easily be copied or
moved from one collection to another, even across server boundaries
and naturally documents can also be edited. Hyperlinks in documents
are created very simply using the mouse.

To control the presentation of collections in Hyper-G the author can
add Attributes like presentation hints, sort order, sequence number
and some more to documents.

Also available for Hyper-G are tools dealing with mass data such as
inserting, deleting, changing or replacing huge numbers of documents
in a single batch job as well as tools for mirroring data on other
servers.

\section{Conclusion}

Hyper-G is designed to manage huge amounts of information and provide
mechanisms making it easy for the information provider to administrate
a server as well as it is designed to make the Internet surfer's life
easy. Integrated sophisticated search mechanisms as well as new
navigation facilities such as a 3D landscape view of the database
strongly ease data retrieval. Hyper-G is a first implementation of the
second generation of hypermedia systems whose features nobody will
want to miss in the future.


\begin{thebibliography}{99999}
\bibitem[Alberti et al 1992]{} Alberti,B., Anklesaria,F., Lindnder,P.,
  McCahill,M., Torrey,D.; Internet Gopher Protocol: A Distributed
  Document Search and Retrieval Protocol; FTP from
  boombox.micro.umn.edu, in directory pub/gopher/gopher\_protocol.
\bibitem[Andrews et al 1994]{} Andrews,K., Kappe,F.; Soaring Through Hyperspace:
  A Snapshot of Hyper-G and its Harmony Client; Proc. of Eurographics
  Symposium on Multimedia/Hypermedia in Open Distributed
  Environments, Graz (1994).
\bibitem[Berners-Lee et al 1994]{} Berners-Lee,T., Cailliau,R., Luotonen,A.,
  Nielsen,H., Secret,A.; The World-Wide Web; Communications of the
  ACM 37,8 (1994), 76-82.
\bibitem[Kappe et al 1993]{} Kappe,F., Maurer,H., Scherbakov,N.; Hyper-G -- A
  Universal Hypermedia System; Journal of Educational Multimedia and
  Hypermedia 2,1 (1993), 39-66.
\bibitem[Maurer 1994]{} Maurer,H.: Advancing the ideas of World-Wide Web;
  Proc. Conf. on Open Hypermedia Systems, Honolulu (1994), 201-203.

\end{thebibliography}

\end{document}
